\documentclass[arxiv]{ucbthesis}
\usepackage[utf8]{inputenc}
\usepackage[T1]{fontenc} 

\usepackage{geometry}
\usepackage[english]{babel}
\usepackage{csquotes}
\usepackage[protrusion=false, expansion=true,factor=1100,final,tracking=true,kerning=true,spacing=true,stretch=40,shrink=10, babel=true]{microtype}
\microtypecontext{spacing=nonfrench}
\SetExtraKerning[unit=space]
    {encoding={*}, family={bch}, series={*}, size={footnotesize,small,normalsize}}
    {\textendash={800,800}, % en-dash, add more space around it
     "28={ ,150}, % left bracket, add space from right
     "29={150, }, % right bracket, add space from left
     \textquotedblleft={ ,150}, % left quotation mark, space from right
     \textquotedblright={150, }} % right quotation mark, space from left
\SetTracking{encoding={*}, shape=sc}{40}
\microtypesetup{protrusion=false}
% \usepackage[T1]{fontenc}
% \usepackage[bitstream-charter]{mathdesign}

\usepackage{biblatex}
\usepackage{pgffor}
\usepackage{appendix}
\usepackage{color}
\usepackage{nicefrac}
\usepackage{graphicx, tabularx}
\usepackage{physics}
\usepackage{amsmath, amssymb, amsthm}
\usepackage[ruled,vlined]{algorithm2e}
\usepackage[capitalise]{cleveref}
\usepackage[hidelinks]{hyperref}

\definecolor{darkred}{rgb}{1.0,0.1,0.1}
\definecolor{darkgreen}{rgb}{0.1,0.7,0.1}
\newcommand{\kd}[1]{\textbf{\color{darkred}[#1 --kd]}}
\newcommand{\bn}[1]{\textbf{\color{darkgreen}[#1 --bn]}}
\newcommand{\inv}{^{-1}}
\newcommand{\R}{\mathbb{R}}
\newcommand{\qt}{\widetilde{q}}
\newtheorem{theorem}{Theorem}[section]
\renewcommand{\[}{\begin{equation}}
\renewcommand{\]}{\end{equation}}
\newcolumntype{M}{>{\(\displaystyle}c<{\)}}

% To compile this file, run "latex thesis", then "biber thesis"
% (or "bibtex thesis", if the output from latex asks for that instead),
% and then "latex thesis" (without the quotes in each case).

% Double spacing, if you want it.  Do not use for the final copy.


% If the Grad. Division insists that the first paragraph of a section
% be indented (like the others), then include this line:
\usepackage{indentfirst}

\addtolength{\abovecaptionskip}{\baselineskip}

\bibliography{references}


\begin{document}

% Declarations for Front Matter

\title{Machine Learning Methods for Cross Section Measurements}
\author{Krish Desai}
\degreesemester{Spring}
\degreeyear{2025}
\degree{Doctor of Philosophy}
\cochairs{Doctor Benjamin Nachman}{Professor Uros Seljak}
\othermembers{Professor Joshua Bloom \\
  Professor Saul Perlmutter}
% For a co-chair who is subordinate to the \chair listed above
% \cochair{Professor Benedict Francis Pope}
% For two co-chairs of equal standing (do not use \chair with this one)
% \cochairs{Professor Richard Francis Sony}{Professor Benedict Francis Pope}
\numberofmembers{4}
% Previous degrees are no longer to be listed on the title page.
% \prevdegrees{B.A. (University of Northern South Dakota at Hoople) 1978 \\
%   M.S. (Ed's School of Quantum Mechanics and Muffler Repair) 1989}
\field{Physics}
% Designated Emphasis -- this is optional, and rare
% \emphasis{Colloidal Telemetry}
% This is optional, and rare
% \jointinstitution{University of Western Maryland}
% This is optional (default is Berkeley)
% \campus{Berkeley}

% For a masters thesis, replace the above \documentclass line with
% \documentclass[masters]{ucbthesis}
% This affects the title and approval pages, which by default calls this
% document a "dissertation", not a "thesis".

\maketitle
% Delete (or comment out) the \approvalpage line for the final version.
%\approvalpage
\copyrightpage

\begin{abstract}
    Precise differential cross section measurements are indispensable for testing Standard Model predictions at the energy frontier and searching for new physics, yet their extraction from collider data is an ill--posed inverse problem.
    %
    \emph{Unfolding} (also known as deconvolution) is the process of removing detector distortions to reconstruct particle--level truth from detector--level data.
    %
    Conventional histogram--based, binned unfolding techniques introduce artifacts, impose arbitrary bin edges, and become computationally prohibitive in high--dimensional phase spaces, potentially obscuring underlying physics.

    This dissertation develops a unified framework that leverages modern machine learning techniques to surmount these limitations.
    %
    First, Neural Posterior Unfolding demonstrates how conditional normalizing flows can serve as differentiable surrogates of detector response, enabling likelihood--based unfolding through implicit regularization.
    %
    Building on this foundation, Moment Unfolding directly extracts distribution moments without binning, providing precise experimental predictions for effective field theories and phenomenological models.
    %
    The framework is further advanced by Reweighting Adversarial Networks (RANs), which perform full spectral unfolding using adversarial training to implement particle--level reweighting steered by detector--level classifiers, offering theoretical and computational advantages over iterative methods.
    
    A critical statistical analysis of event correlations in unfolded data reveals systematic misestimation of uncertainties when these correlations are ignored, leading to methodological recommendations that ensure correct coverage for all derived observables.
    %
    The methods presented are validated using both idealized Gaussian distributions and proton--proton collision simulations of the CMS experiment as realistic particle physics examples, specifically simulations of Z+jets events, demonstrating significant improvements in precision, accuracy, and computational efficiency, achieving orders of magnitude speed up while maintaining unbiased recovery of sharp spectral features.

    By marrying statistical rigour with powerful machine learning methods, this work establishes a scalable blueprint for precision measurements at current and future colliders.
    %
    The resulting open--source software enables more reliable extraction of fundamental physics parameters from complex detector data, advancing our ability to test theoretical models and potentially discover new phenomena in high energy physics experiments.
\end{abstract}
\chapter*{Publications Resulting from This Thesis}
\begin{refsection}
\nocite{NeurIPS.2021, PhysRevD.105.096031, desai2022deconvolving, desai2024moment, acosta2024npu, zhu2024multidimensional, Desai:2025mpy}
\printbibliography[heading=none]
\end{refsection}

\begin{frontmatter}

\begin{dedication}
\null\vfil
\begin{center}
To name\\\vspace{12pt}
text
\end{center}
\vfil\null
\end{dedication}

% You can delete the \clearpage lines if you don't want these to start on
% separate pages.

\tableofcontents
\clearpage
\listoffigures
\clearpage
\listoftables

\begin{acknowledgements}
Acknowledgements

\end{acknowledgements}

\end{frontmatter}

\pagestyle{headings}

% (Optional) \part{First Part}


\foreach \i in {1,...,11} {
    \include{chapters/chap\i}
}

\printbibliography

\appendix
\foreach \i in {1, 2, 3}{
    \include{appendices/app\i}
}
\end{document}
