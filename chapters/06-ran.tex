\chapter{RAN: Reweighting Adversarial Networks}
\label{chap:ran}
\section{The need for full spectral measurements}
    \label{sec:need-for-density-unfolding}
    \subsection{Motivation: Beyond Moments and Binned Spectra}
    Measurements in high energy physics traditionally report differential cross sections in a binned format, or even just a few summary statistics (moments) of a distribution. 
    %
    This approach has produced numerous important results in the past\kd{cite}, but it fundamentally limits the information available for theoretical interpretation.
    %
    A small set of moments (e.g. mean, variance) offers only a coarse summary of a probability distribution and can hide critical features.
    %
    In fact, infinitely many different distributions can share the same first few moments.
    %
    Two observables with identical mean and variance may have drastically different tails or multi-modal structures that only a full distribution reveal.
    %
    Thus, relying solely on low--order moments risks missing new physics signals or subtle QCD effects that manifest as shape differences rather than overall normalization changes.

    Binning an observable into histograms is a more detailed approach than just quoting moments, but it too imposes significant limitations.
    %
    Finite bin widths smear out fine structures and impose an arbitrary discretization on inherently continuous spectra.
    %
    Moreover, once data are binned, it becomes impossible to reconstruct or analyze the distribution for any arbitrary transformation of that observable without returning to the original unbinned data.
    %
    For example, if a cross section is unfolded in bins of an angle $\theta$, one cannot later obtain the distribution in $\cos\theta$ or $\ln\theta$ except by repeating the unfolding from scratch.
    %
    In contrast, an unbinned (or “full spectral”) measurement preserves maximum information, allowing \textit{a posteriori} reprocessing such as deriving moments, re-binning in different intervals, or studying functions of the measured observable.
    %
    This flexibility is especially valuable when comparing with various theoretical models, each of which may suggest different variables or summary statistics to highlight.

    Another key motivation for capturing the \emph{full differential spectrum} is that many theoretical predictions in quantum chromodynamics (QCD) and beyond are sensitive to the detailed shape of distributions.
    %
    New physics might appear as excess events in the tails of kinematic distributions, or subtle distortions across a spectrum rather than an overall rate change.
    %
    Precision QCD studies often rely on the \emph{scaling patterns} of entire distributions with energy or other parameters, not just their averages.
    %
    By measuring the full spectrum, experimental results can be directly fed into global fits or theory calculations that integrate over the entire phase space.
    %
    In summary, to fully exploit the data and enable the broadest possible comparisons to theory, it is imperative to go beyond a few moments or fixed--bin histograms and aim to unfold the continuous differential cross section itself.

    Let us now discuss in more detail some of the central limitations of binned approaches and approaches focused on unfolding a few moments summarised above in greater detail.
    \begin{itemize}
        \item \textbf{Information Loss:}
            Low--order moments (such as mean or variance) condense an entire distribution into one or two numbers.
            %
            This sacrifices information about higher--order fluctuations and tail behavior.
            %
            Many distinct distributions can reproduce the same set of moments, so important differences (e.g. a long tail vs. a sharp cutoff) remain hidden if only moments are reported.
            %
            Even in cases where theory predicts moments more readily than spectra (as in some QCD calculations of energy scaling of moments\kd{cite}), relying exclusively on moments means discarding data that could otherwise constrain models.
            %
            Binned histograms similarly integrate the underlying distribution over each bin, blurring details smaller than the bin width.
            %
            Fine binning could, in principle, mitigate this, but the detector resolution and statistical limits cause too--fine bins to lead to instability due to bin migration effects \kd{cite} and large uncertainties.
        \item \textbf{Binning Artifacts and Biases:}
            Any choice of bin boundaries is inherently arbitrary.
            %
            Two experiments measuring the same observable might choose different binning schemes, complicating direct comparisons.
            %
            Small shifts in bin edges can redistribute events and lead to apparent differences that are purely due to binning choices rather than physics.
            %
            Furthermore, when binning multivariate data (or examining an observable’s moments in bins of another quantity), the necessary discretization in each dimension can introduce artificial discontinuities.
            %
            These artifacts hinder comparing unfolded results with continuous theoretical predictions or with results from other experiments that used different bin definitions\kd{cite}.
            %
            Additionally, binning can bias downstream analysis;
            %
            for instance, extracting moments from a binned distribution requires assuming a shape within each bin (often a constant or linear interpolation).
            %
            If the true distribution varies non--linearly inside the bin, the extracted moment is biased by the bin size and shape assumption such that even the sign of the error cannot be known \textit{a priori}.\kd{cite Moment Unfolding}
            %
            An unbinned measurement eliminates this intermediate step and potential bias.
        \item \textbf{Limited Dimensionality:}
            Perhaps most importantly, binned unfolding severely constrains the number of observables one can unfold simultaneously.
            %
            Each additional dimension (feature) requires exponentially more bins to maintain a given resolution.
            %
            In practice, traditional unfolding is often performed in one dimension at a time (or at most two) because higher--dimensional histograms would be too sparsely populated.
            %
            This precludes measuring cross sections differential in many variables at once.
            %
            It also means that if one is interested in an observable $O$ that is a complicated function of several kinematic quantities, one either must unfold those quantities jointly with fine binning (which is infeasible) or settle for unfolding a projection of $O$ in one dimension (which loses information).
            %
            In contrast, unbinned approaches can handle high-dimensional data naturally, since they do not require constructing an \(d-\)dimensional grid of bins.
            %
            In principle, unfolding the full phase space (i.e. a fully differential cross section in all relevant kinematic variables) is only realistic with an unbinned strategy.
    \end{itemize}        
    Table~\ref{tab:meas_compare} summarizes these differences between unbinned moment--based unfolding, binned density unfolding, and unbinned full--spectrum unfolding methods.
    %
    By construction, a full spectral measurement retains maximal information and avoids discretization issues, at the cost of a more challenging unfolding procedure.
    %
    These challenges, as discussed next, were a major deterrent historically, explaining why experiments long favoured simplified (binned or moment) results despite their limitations.
    
    \begin{table}
        \centering
        \caption{Comparison of measurement approaches.
        %
        Reporting only low--order moments loses most distribution information.
        %
        Binned differential cross sections retain shape information but suffer from discretization and limited dimensionality.
        %
        Full spectral measurements preserve the complete distribution, enabling maximal reusability and detailed theory comparisons, but require regularizing a much more ill--posed training.
        }
        \label{tab:meas_compare}
        \begin{tabular}{lccc}
            \toprule
            \textbf{Aspect} & \textbf{Moments only} & \textbf{Binned spectrum} & \textbf{Full spectrum} \\
            \midrule
            Information           & Low   & Moderate & High \\
            Reuse         & High      & Limited  & High \\
            Dimensions    & Many       & 1 to 2   & Many \\
            Stability              & Easy & Moderate & Hard \\
            \bottomrule
        \end{tabular}
    \end{table}

    
    \subsection{Use Cases}
    Some of the most compelling motivations for unbinned unfolding unfolding of probability density functions come from specific classes of observables and analyses in high--energy physics that demand detailed distributions rather than summary measures.
    %
    A few examples are
    \begin{itemize}
        \item \textbf{Jet substructure observables:}
            %
            Modern studies of jet physics often examine intricate internal properties of jets, for example, the distribution of jet mass, angularity, $N-$subjettiness ratios like $\tau_{21}$, energy correlation functions, and so on.
            %
            These observables have rich distributions shaped by QCD radiation and hadronization inside the jet.
            %
            For instance, the jet mass spectrum in hadronic collisions exhibits a steeply falling shape with resonance peaks or grooming--induced features in certain regions;
            %
            capturing these details is essential for validating parton shower models and grooming techniques.
            
            If only the average jet mass or a few quantiles were reported, one would miss the full story of how often jets are heavy vs. light, or how substructure techniques sculpt the distribution.
            %
            Similarly, distributions of $\tau_{21}$ (a ratio used to tag two--prong substructure) contain information about the fraction of jets with two subjets vs. one, which is crucial for signal (e.g. boosted $W$) vs background (QCD jet) discrimination.
            %
            Capturing the entire $\tau_{21}$ spectrum allows experimentalists and theorists to identify where in the distribution their models agree or fail, rather than just comparing an efficiency at a fixed cut.
            %
            Unfolding these jet substructure distributions in an unbinned way provides a high--resolution view of QCD dynamics and is increasingly necessary as theory tools (like analytic resummation or first--principles simulation) improve to the point of predicting differential shapes\kd{cite}.
            %
            In fact, recent measurements have demonstrated the power of full--phase--space unfolding for jets, using multivariate ML techniques to correct detector effects and obtain particle--level jet observable spectra without binning\kd{cite}.
            %
            These cases underscore that jet physics benefits enormously from preserving the full shape information.
        \item \textbf{Charge and flavour sensitive observables:}
            %
            Some observables aim to distinguish particle charge or flavour inside complex final states, and their distributions can be especially telling.
            %
            An example is the jet charge distribution, the electric charge of a jet computed from the momentum--weighted charges of its constituent particles.
            %
            This quantity is used to tag whether a jet originated from a quark of a given electric charge (as opposed to a gluon, which has none).
            %
            The jet charge is a continuous-valued observable that can take positive and negative values, and experiments measure its probability distribution for jets of various momenta.
            %
            The full shape of the jet charge distribution contains information about the underlying quark/gluon mixture and fragmentation processes;
            %
            for instance, a broader distribution indicates a mix of high--charge (quark--origin) and zero--charge (gluon--origin) jets, whereas a narrow distribution peaked near zero suggests predominantly gluon jets.
            %
            Only by unfolding the entire jet charge spectrum (including its dependence on jet $p_T$) can one provide data precise enough to tune fragmentation models and compare to theoretical calculations of charge transport in jets\kd{cite}.
            
            Another example is the distribution of identified particle multiplicities (e.g. number of charged hadrons in an event or in a jet).
            %
            The multiplicity distribution is discrete but often measured as a histogram.
            %
            It is a particularly challenging observable to unfold because it is strongly sensitive to soft QCD and hadronization effects.\kd{cite}
            %
            Two Monte Carlo generators might predict the same average multiplicity but differ in the width or tail of the multiplicity distribution, which affects extreme cases (like very high multiplicity events).
            %
            Such observables are often \emph{infrared--unsafe} theoretically (because soft particle emission has no cutoff, perturbative predictions diverge for the distribution shape), meaning theory must rely on phenomenological models.
            %
            The only way to constrain and improve those models is for experiments to provide the unfolded full distributions of these IR--unsafe observables at particle level.
            %
            In summary, any analysis where internal structure, charge assignments, or other detailed event properties matter will benefit from (or even require) full spectral unfolding rather than a few summary numbers.
        \item \textbf{Multi--differential and high--dimensional measurements:}
            The ultimate form of “full” spectral measurement is unfolding in multiple kinematic dimensions simultaneously, effectively measuring a multi--differential cross section over a high--dimensional phase space.
            %
            While this is extremely challenging, certain physics questions demand correlating several observables.
            %
            For example, consider measuring an observable $O$ as a function of another variable $Q$ (say, the distribution of jet substructure variable $O$ in bins of jet $p_T$ or event energy $Q$).
            %
            With traditional methods, one would perform a two--dimensional unfolding (bins in $O$ vs bins in $Q$) to then extract moments or other features as a function of $Q$.\kd{cite Moment Unfolding}
            %
            Binning in two (or more) dimensions quickly suffers from sparse data in many bins and complicated systematic uncertainties.
            %
            An unbinned approach, by contrast, could in principle unfold the joint $(O,Q)$ distribution without requiring an explicit grid, allowing arbitrary slicing and analysis after the fact.
            %
            This is especially relevant in the era of high--luminosity colliders, where huge datasets invite more differential measurements.
            %
            Rather than publish dozens of separate one--dimensional spectra (each in a single kinematic region), one could publish a multi--dimensional unfolded distribution that coherently captures all correlations.
            %
            Such a result would be far more powerful for global interpretations, albeit significantly more complex to obtain.
            %
            Unbinned density unfolding techniques are a step toward this ambitious goal, enabling higher dimensional unfolding than previously feasible with manageable uncertainties.
    \end{itemize}
    \subsection{Challenges in Unfolding Full Distributions}
        The push for unbinned, high--resolution unfolded spectra comes with substantial challenges.
        %
        Unfolding a full distribution (especially in multiple dimensions) is a markedly more ill--posed problem than unfolding a small set of summary statistics or coarse bins.
        %
        These challenges are statistical, physical, and computational.

        At its core, unfolding requires inverting the detector response---a many--to--many mapping where a given particle--level distribution can produce a range of detector--level outcomes due to resolution and inefficiencies.
        %
        This inversion is ill--posed, that is to say, infinitely many particle--level spectra are, in principle, consistent (within uncertainties) with a given set of detector--level data, especially if the data are treated in fine detail.
        %
        Small fluctuations or statistical noise in the detector--level histogram can cause huge oscillations in the naive unfolded solution if one attempts a direct inversion of the response matrix.
        %
        In classical unfolding methods, this is well known.
        %
        Directly inverting the response matrix $\mathbf{R}$ leads to amplified noise and unstable solutions.\kd{cite}
        %
        The problem is exacerbated when the “matrix” is essentially continuous (unbinned) since here one is trying to reconstruct a function rather than a finite vector, which has infinitely many degrees of freedom.
        %
        Without any constraints or regularization, unfolding is mathematically underdetermined; one must introduce additional information to obtain a physical solution.
        %
        This additional information can be statistical regularization (penalizing roughness in the solution), or a strong prior (initial guess of the spectrum) to guide the result.
        %
        Either way, the unfolded distribution will depend to some degree on these assumptions, which is a point of concern.

        All unfolding methods require some initial model or ansatz for the true distribution, explicitly or implicitly.
        %
        Traditional regularized unfolding (e.g. Bayesian iterative methods or Tikhonov regularization) starts from a prior distribution and updates it in light of the data.
        %
        If the prior is significantly wrong in a region where data have low sensitivity, the unfolded result may inherit that wrong shape (bias) because there isn’t enough information in the data to correct it.
        %
        For binned methods with strong regularization, the unfolded spectrum can end up looking very much like the prior except in regions where the data clearly indicate otherwise.
        %
        When unfolding a full spectrum, prior dependence can be even trickier: with many bins or continuous degrees of freedom, there are more opportunities for the prior assumptions to creep in unless the method explicitly works to mitigate this.
        %
        Modern ML based unbinned unfolding techniques like iterative reweighting (used by \textsc{\textsc{OmniFold}}) attempt to reduce prior bias by gradually adjusting the prior to fit the data\kd{cite}, but they still require that the initial simulation populate all regions of phase space that data might cover.\kd{cite}
        %
        In other words, if the true distribution has support outside the domain of the starting simulation, no unfolding can recover that---a condition that applies to all known methods.
        %
        This need for sufficient overlap in the support of the prior and true distributions is a fundamental mathematical limitation: full spectral unfolding demands that our simulation model is flexible and broad enough to encompass the truth, at least roughly, or else certain features will be missed entirely.

        Unfolding more finely (or in more dimensions) inherently means extracting more parameters (or effectively, more bins worth of information) from the same finite dataset.
        %
        This trade--off means individual elements of the unfolded spectrum will have larger statistical uncertainties than if the data were aggregated into a few bins or moments.
        %
        For example, unfolding 100 bins will yield each bin count with larger uncertainty than if one had combined them into 10 bins.
        %
        If not carefully handled, an unbinned unfolding will overfit statistical fluctuations in the observed data (referred to as \emph{sculpting} in the machine learning literature\kd{cite}.)
        %
        Effective regularization is essential to prevent noise amplification.
        %
        Additionally, the detector response kernel must be well modelled.
        %
        Any mismodeling (systematic error) can imprint itself on the unfolded result in complex and unpredictable ways.
        %
        When only a few numbers are extracted, one can sometimes correct for known detector biases by simple scale factors; but when unfolding an entire distribution, any mismodeling of the response shape can distort the unfolded spectrum non--uniformly.
        %
        This places high demands on detector simulation fidelity and on methods to incorporate systematic uncertainties (e.g. variations of the response model) into the unfolding procedure.
        %
        Fully Bayesian approaches or profile likelihood methods can propagate uncertainties to the unfolded spectrum, but doing so in high dimensions is computationally intensive.
        %
        In short, the richer the information we unfold, the more careful we must be to quantify the reliability of each feature of the spectrum.

        Traditional unfolding algorithms (like solving $\mathbf{R}^{-1}\vb*{\mu}$ for binned histograms) scale poorly as the number of bins grows.
        %
        These traditional algorithms like iterative Bayesian unfolding (IBU, or D’Agostini method) are relatively fast for tens of bins but become slow for hundreds of bins, and practically unusable for thousands of bins or continuous data, as each iteration must refine a fine--grained distribution.
        %
        Unbinned algorithms, on the other hand, typically rely on machine learning or Monte Carlo sampling and are computationally expensive: they may involve training complex models on large datasets or performing high--dimensional optimizations.
        %
        For instance, iterative reweighting methods train a classifier multiple times (each iteration is a full supervised learning task on the dataset)\kd{cite}, and generative approaches might require training a high--capacity generative model flow on millions of events.
        %
        These computations require substantial computing resources (CPU/GPU), careful hyperparameter tuning, and sometimes suffer from convergence issues.
        %
        Ensuring that an unbinned unfolding converges to a stable solution without excessive computation is a non--trivial challenge.
        %
        In summary, the practical feasibility of unfolding full spectra depends on advances in algorithms and computing, precisely the advances that modern machine learning--based methods aim to provide.
        
        Despite these challenges, the drive to measure full spectra is strong because of the scientific payoff.
        %
        The next subsection discusses how recent machine learning methods, including the RAN architecture, tackle these issues.

    \subsection{Unbinned Unfolding Approaches}
        Recent years have seen rapid development of machine learning techniques for unfolding, which seek to overcome the limitations of traditional methods and make full spectral measurements possible in practice.
        %
        Broadly, these approaches fall into two categories, those based on \textit{reweighting} an existing simulated sample to better agree with data, and those based on \textit{generating} new events (typically with generative models) to reproduce the data distribution.
        %
        Both categories strive to avoid fixed histogram binning and instead work with unbinned data, using the power of ML to handle high--dimensional inputs and complex detector responses.

        One of the pioneering ML-based methods is \textbf{\textsc{\textsc{OmniFold}}}\kd{cite}, which introduced an unbinned, multivariate unfolding technique using iterative reweighting.
        %
        \textsc{\textsc{OmniFold}} uses classifier neural networks to reweight a Monte Carlo sample.
        %
        Essentially, it trains a classifier to distinguish between data and simulation, and then uses the classifier output to assign weights to simulation events such that the weighted simulation better matches the data.
        %
        This procedure is done in stages (iterations) and at both detector and particle level in alternation.\kd{cite}
        %
        After several iterations, the method produces a set of weights for generation events that yield an unfolded distribution.
        %
        \textsc{\textsc{OmniFold}} demonstrated, for the first time, that one can simultaneously unfold many observables (even the entire event record) without binning, given a sufficiently flexible classifier and a robust iterative scheme.\kd{}
        %
        It has been successfully applied in multiple experimental analyses,\kd{}.
        %
        The key advantages of \textsc{\textsc{OmniFold}} are that it naturally accounts for high--dimensional correlations (since the classifier can use any or all features) and it actively mitigates prior dependence by iterative refinement, effectively performing a form of expectation--maximization to find a self consistent unfolded result that does not overly rely on the initial generation distribution.
        %
        However, a noted downside is its computational cost: each iteration requires training two classifiers to convergence, and in realistic cases one might need $5 - 10$ iterations, amounting to training a large number of networks.
        %
        This iterative nature can also complicate uncertainty evaluation and hyperparameter tuning (e.g. one must subjectively choose when to stop iterating to avoid instability).
        %
        Still, \textsc{\textsc{OmniFold}} set the stage for practical unbinned unfolding and proved the feasibility of full spectral measurements on real collider data.

        In parallel, other methods have explored using explicit generative models to unfold distributions.
        %
        For example, VAEs have been employed to learn a mapping from random noise to particle--level distributions such that, when those events are passed through a detector simulation, the output matches the observed data distribution\kd{cite}.
        %
        Similarly, normalizing flows and other invertible neural networks have been used to model the probability density function of true observables and deform it until its convolution with the detector response matches data\kd{cite}.
        %
        These approaches are extremely powerful in principle: a sufficiently flexible generator could capture the full true distribution without needing a starting simulation.
        %
        In practice, however, training such generative models is challenging
        %
        Pure generative unfolding would require vast amounts of data to constrain the high--dimensional space and can suffer from mode dropping (the generator might fail to produce some less common features of the distribution if not properly incentivized).
        %
        Additionally, learning a full generative model from scratch means the method must learn \emph{both} the underlying physics distribution \emph{and} compensate for detector effects at the same time.
        %
        This is a high--dimensional optimization that can be unstable or require careful conditioning.
        %
        Some recent works have attempted to combine generative modelling with explicit usage of the known detector response to guide the training (for instance, using differentiable detectors or gradient-based deconvolution\kd{cite}), but these are still at the experimental stage.

        Between pure reweighting (which uses an existing simulation as a starting point) and pure generation (which starts from random noise) lies an attractive compromise: use machine learning to \emph{reweight or recalibrate} an initial simulation in a single pass, by directly comparing its predictions to data.
        %
        The \textbf{Reweighting Adversarial Networks (RAN)} method falls into this category.
        %
        Other examples include optimal transport inspired methods that train a single neural network to learn a transport map by minimizing some distance between weighted simulation and data distributions\kd{cite}.
        %
        Such approaches leverage the fact that modern simulations are a good but imperfect approximation of reality; rather than throw them away and learn from scratch, it may be easier to learn small corrections (reweightings) to the simulation.
        %
        In this sense, RAN and similar methods treat unfolding as a density ratio estimation problem: find a function $g(z)$ such that the weighted MC distribution $g(z)\,q(z)$ matches the true distribution $p(z)$.
        %
        If $Z$ denotes particle--level kinematics, this $g(z)$ effectively encapsulates how the Monte Carlo needs to be modified to agree with nature.
        %
        The challenge is that we cannot observe $p(z)$ directly---we only have its smeared version at detector level.
        %
        Therefore, these algorithms set up a {two-level} training objective to adjust $g(z)$ based on how well the weighted events, after going through the detector simulation, agree with the detector--level data.
        %
        This two-level problem is naturally addressed with adversarial setups (a simulator or reweight function competing against a discriminator) or with optimal transport objectives that connect particle and detector level distributions.

        In summary, the landscape of modern unfolding methods includes iterative classifiers (\textsc{\textsc{OmniFold}}) \kd{cite}, single--shot adversarial reweighting (e.g. RAN), and generative approaches (cINNs, VAEs)\kd{cite}, each with their own strengths.
        %
        Table~\ref{tab:unfold_methods} provides a high-level comparison.
        %
        All these methods aim to enable unbinned, high-dimensional unfolding; they chiefly differ in how they incorporate prior knowledge, whether they iterate, and how computationally intensive they are.
        %
        Notably, most current ML--based methods (except full generative ones) still require that the starting simulation has support in the regions of interest (the issue of support overlap).
        %
        They also share common challenges of training stability and overfitting prevention, which are addressed through techniques like regularization or specific loss functions (for instance, \textsc{OmniFold} uses early stopping of iterations, while adversarial methods use regularized discriminators).

        \begin{table}
            \centering
            \caption{Comparison of unfolding methods. 
            %
            “Unbinned” indicates the method can use continuous data without fixed histograms.
            %
            “Iterative” indicates whether multiple training iterations are required (not counting training epochs, which all neural methods require internally).
            %
            “Perturbative” indicates that the method reweights a MC sample (as opposed to generating events from scratch).
            %
            RAN (this work) occupies a middle ground, using a prior simulation like \textsc{OmniFold} but aiming to avoid costly iterations by employing a stable adversarial training objective.}
            \label{tab:unfold_methods}
            \begin{tabular}{lccc}
                \toprule
                \textbf{Method} & \textbf{Unbinned} & \textbf{Iterative} & \textbf{Uses sim.} \\
                \midrule
                Traditional (IBU, TUnfold) & No & Sometimes & Yes\\
                Discriminative (\textsc{OmniFold})      & Yes & Yes & Yes \\
                Generative (cINNs/VAEs)             & Yes & No  & No (from noise)\\
                Adversarial reweight (RAN)        & Yes & No  & Yes \\
                \bottomrule
            \end{tabular}
        \end{table}

    \subsection{Addressing Key Challenges}
        The Reweighting Adversarial Networks (RAN) approach described in this chapter is designed to tackle the aforementioned challenges head--on, enabling full spectral unfolding with improved stability and efficiency.
        %
        Here I outline how the methodology of RAN addresses the needs and difficulties detailed above (a complete description of the method follows in later sections, so the focus here is on concepts rather than implementation details).
        
        RAN explicitly targets the entire distribution rather than a fixed set of moments.
        %
        In fact, it can be viewed as an extension of the Moment Unfolding method\kd{cite} described in Chapter~\ref{chap:moment-unfolding} to infinitely many moments.
        %
        By using a flexible neural network to parametrize the weighting function, RAN does not impose a rigid functional form limited to a few moment constraints.
        %
        This means it has the capacity to adjust the simulation such that not only the first few moments, but \emph{all} features of the distribution (in principle, every differential element) are brought into agreement with the observed data.
        %
        Framing the problem this way ensures that when RAN converges, the result is a faithful unfolded spectrum, from which moments or any other summary statistic can be derived.
        %
        Importantly, because it handles the full spectrum at once, RAN avoids the bias of selecting certain observables upfront---it lets the data inform the entire shape. 
        %
        Hence RAN is aligned with the basic motivation of full spectral measurements: nothing gets thrown away or averaged out prematurely.

        By construction, RAN is an unbinned method.
        %
        It operates on individual events, using distance measures defined on samples rather than comparing binned counts.
        %
        The adversarial training framework means that a discriminator network looks at distributions of features (at detector level) and tries to tell apart weighted simulation from real data.
        %
        If it finds any discrepancy, no matter how localized, the reweighting network is encouraged to adjust weights to eliminate that discrepancy.
        %
        This is effectively a fine--grained comparison across the full phase space, without ever projecting data into predetermined bins.
        %
        As a result, RAN does not suffer from the bin alignment issues or interpolation biases that plague binned unfolding.
        %
        Different experiments using RAN on the same observable should, in principle, get comparable results without worrying that one used 50\% purity bins and another used 70.7\% purity bins, since neither uses bins at all.
        %
        The output of RAN can be presented as a smooth distribution or as a weighted event sample at particle level, which can then binned or analysed downstream as needed.
        %
        This flexibility maximizes the utility of the measurement.

        One of the core design goals of RAN is to eliminate the need for multiple iterative reweighting cycles (as in \textsc{OmniFold}).
        %
        Instead of training sequential classifiers for each iteration, RAN uses a single coupled training procedure where the particle--level reweighting function and the detector--level discriminator are learned together (analogous to a generator and critic in a Wasserstein GAN).
        %
        This yields a one--shot solution for the weights after convergence.
        %
        In practice, this means RAN trains one neural network (the reweighter) with feedback from another (the discriminator) in one integrated run.
        %
        The computational cost is roughly equivalent to training a single GAN rather than a dozen separate classifiers.
        %
        As demonstrated in Sec.~\kd{ref RAN results}, this can lead to a significant reduction in runtime and resource usage for large--scale unfolding tasks\kd{cite}.
        %
        Furthermore, by using an optimal transport--based loss (inspired by the Wasserstein GAN) and techniques like spectral normalization in the discriminator, RAN achieves stable training dynamics.\kd{cite}
        %
        The Wasserstein metric provides a smooth loss function that correlates well with distribution difference, avoiding the chaotic or oscillatory behaviour that vanilla GANs can exhibit, described in Sec~\ref{subsubsec:GANs}.
        %
        Spectral normalization bounds the discriminator’s gradient, effectively regularizing the learning and preventing mode--collapse or instability.\kd{cite}
        %
        These choices were crucial technical innovations needed to extend the moment--matching method to a full-spectrum matching method.\kd{cite}
        %
        The end result is that RAN converges reliably to a solution where the weighted generation agrees with truth, all in a single training loop.
        %
        This addresses the computational challenge by trading an iterative series of simpler trainings for one more complex adversarial training, which has been shown to be tractable and efficient.

        Like any reweighting--based approach, RAN does require a MCMC samples and will fail if the true distribution lies entirely the support of the generation.
        %
        However, RAN upon convergence, provably does not rely on the exact prior shape within the supported region.
        %
        The use of an adversarial loss means RAN is effectively solving a constrained optimization to find the weight function that makes the simulation statistically indistinguishable from data by reweighting only at particle level.
        %
        RAN’s connection to the Boltzmann entropy--inspired moment unfolding provides a theoretical understanding of how it approaches the “maximum entropy” solution consistent with the constraints (the data)\kd{cite}.
        %
        This is desirable because the maximum entropy solution is the least biased one given the information at hand.
        %
        In essence, RAN inherits the moment method’s stability from having a limited functional form, and compensates for the larger freedom of full spectra by adding proper regularization in training, thereby taming the ill--posedness.

        An important practical aspect is that RAN yields physically reasonable unfolded distributions, such as non--negative weights and normalized total cross sections.
        %
        By parametrizing the weight function as $g({z}) = \frac1P\exp(-NN({z}))$ (one convenient choice), we ensure $g({z})\ge 0$ for all events, so the unfolded cross section is positive--definite.
        %
        The training objective can be set up to include a normalization constraint or simply allow the weights to float the total normalization to match the data yield.
        %
        This avoids unphysical outcomes like negative weight factors or mismatched totals that can occur in some matrix inversion methods without extra constraints.
        %
        Moreover, because RAN operates at the level of reweighting actual events, the resulting weighted event sample can be validated easily.
        %
        One can always forward--fold the weighted sample through the detector simulation to check that it indeed reproduces all features of the observed data (closure test).
        %
        This built--in consistency check is a powerful advantage of the reweighting paradigm in general.
        %
        It is straightforward to verify that the full spectral measurement is successful, by confirming that no residual differences remain between data and weighted simulation across all distributions of interest.

        It's worth noting that the RAN framework could potentially be extended to perform background subtraction simultaneously with unfolding by modifying the weight parameterization to allow negative weights.
        %
        This might be particularly valuable for heavy ion physics, where background contamination presents significant challenges.\footnote{
            While this extension has not been implemented or tested in the current work, allowing negative weights could potentially enable joint background subtraction and unfolding.
            %
            This approach might be especially valuable for heavy ion collisions where background separation is notoriously challenging due to the high multiplicity environment and complex underlying events \kd{arXiv:2402.10945}.
            %
            Several studies have explored related joint unfolding and background estimation approaches, including work on machine learning based background subtraction methods that can reduce fluctuations below the statistical limit \kd{Machine Learning based jet momentum reconstruction in heavy-ion collisions, arXiv:2305.16826 , arXiv:2402.10945}.
            %
            The difficult interplay between background subtraction and unfolding has been examined in detail by Apolinário et al.\kd{An analysis of the influence of background subtraction and quenching on jet observables in heavy-ion collisions}, who analyzed how different background subtraction methods affect jet observables in heavy ion collisions.
            %
            Recent research has also improved background subtraction techniques specifically for jet substructure measurements\kd{Improved background subtraction and a fresh look at jet sub-structure in JEWEL}, demonstrating that proper handling of medium response is crucial for meaningful comparison with experimental data
            %
            The theoretical framework of transport theory deconvolution with background contributions \kd{cite} could provide mathematical guidance for such extensions.
            %
            Development of these capabilities would require careful validation and is left as future work.
        }
        In conclusion, RAN is conceived to meet the need for unfolding multivariate probability densities by addressing the key challenges that have historically prevented such measurements.
        %
        It leverages modern ML (specifically adversarial training and neural network flexibility) to unfold complete distributions without binning, while incorporating solutions to the statistical and computational pitfalls (regularization via WGAN, one--pass training, etc.).
        %
        The following sections will describe the RAN methodology in detail and demonstrate its performance.
        %
        Here, we have established \emph{why} a method like RAN is needed---because it enables us to unfold the maximal information from experimental data, the full differential cross section, in a way that is both theoretically and practically sound, thus empowering deeper insights into particle physics phenomena.


\section{From Moments to Complete Differential Cross Section Spectra}
    In Sec.~\ref{sec:need-for-density-unfolding} we discussed the motivation for unfolding complete differential cross sections---retrieving the full distribution of an observable at particle--level from distorted detector--level data.
    %
    We now build a rigorous framework linking distribution moments to the full probability density, and review how leveraging moment constraints can facilitate unbinned unfolding.
    %
    By treating moments as fundamental constraints (as opposed to bin--by--bin values), we establish a pathway from a limited set of summary statistics to a complete unfolded spectrum.
    %
    This section first develops the theoretical foundation connecting moments and probability distributions (including moment--generating functions and maximum entropy arguments), then surveys classical and modern unfolding methods that rely on moment constraints (such as regularization and entropy--based techniques).
    %
    This is followed by a discussion of the advantages and pitfalls of moment--based unfolding, notably issues of ill--posedness and non--uniqueness, and the regularization strategies to mitigate them.
    %
    Finally, I explain how the Moment Unfolding method introduced in Chapter~\ref{chap:moment-unfolding} naturally extends toward unfolding the entire distribution (all moments) via a GAN--like approach inspired by the Boltzmann distribution, setting the stage for a full differential cross section measurement.

    \subsection{Moments and the Full Probability Distribution}
        Statistical moments provide a powerful, coarse--grained description of a probability distribution.
        %
        The $n$th moment (about zero) of a random variable $Z$ is $\mathbb{E}[Z^n]$, and moments about the mean (central moments) characterize the shape (variance, skewness, kurtosis, etc.) of the distribution.
        %
        In principle, an infinite sequence of moments can completely characterize a distribution.
        %
        This is referred to as the moment representation of a distribution.
        %
        If all moments ${\mu_1, \mu_2, \dots}$ are known and certain technical conditions hold (e.g. the moment-generating function exists in a neighbourhood of 0), then there is a unique probability density consistent with those moments.\kd{DOI:10.11647/obp.0333.07}\kd{cite one more}
        %
        The moment generating function (MGF), $M_Z(t) = \mathbb{E}[e^{tZ}]$, encapsulates the entire moment sequence via its Taylor expansion,
        \[
            M_Z(t) \;=\; 1 + \sum_{n=1}^\infty \frac{t^n}{n!}\,\mathbb{E}[Z^n]~,
        \]
        and serves as a proxy for the full distribution.
        %
        In fact, knowledge of $M_Z(t)$ (or the characteristic function) allows one to reconstruct the probability density $p_Z(z)$ by inverse transformation.
        %
        In short, an infinite set of moment constraints is mathematically equivalent to knowing the complete distribution.

        In practice, however, we can only estimate a finite number of moments from data with finite precision.
        %
        A finite moment set underdetermines the distribution, because infinitely many distinct densities can share the same first $n$ moments.
        %
        This non-uniqueness under finite information is closely related to the ambiguity of solutions when attempting to invert detector effects: the data usually provide limited “moments” of the true distribution (for example, binned event counts are themselves integrals of the true spectrum over bin ranges).
        %
        As $n$ increases, the moment constraints become more informative and the space of consistent solutions shrinks, but noise and uncertainties also grow.
        %
        In the limit of $n\to\infty$, the true distribution would be recovered, but this limit cannot be reached exactly with finite and noisy data.

        \subsubsection{The Maximum entropy principle and exponential families}
            Given a few known moments, a common approach to approximate the underlying distribution is to apply the principle of maximum entropy.\kd{ISBN 3-89429-543-0}
            %
            This principle dictates that, among all distributions satisfying the known moment constraints, we should prefer the one with the largest entropy (i.e. the least additional assumptions or information).
            %
            Imposing constraints on expectations $\mathbb{E}[f_a(Z)] = c_a$ (for some set of functions $f_a(Z)$ defining the moments of interest) leads to a unique maximum-entropy solution in the exponential family.
            %
            In particular, one finds a probability density of the form
            \[
                p^*(z) \;=\; \frac{1}{Z(\boldsymbol{\beta})}\,\exp\!\Big(-\sum_{a}\beta_a\,f_a(z)\Big)~,
            \]
            where $\beta_a$ are Lagrange multipliers adjusted to enforce the desired $\mathbb{E}_{p^*}[f_a(Z)] = c_a$, and $Z(\boldsymbol{\beta}) = \int \exp(-\sum_a \beta_a f_a(z))\,\dd z$ is the normalization factor (partition function).
            %
            This is analogous to the Boltzmann distribution in statistical mechanics, which maximizes entropy given a fixed average energy.
            %
            Indeed, if we choose $f_a(z) = z^a$ (the monomials), the above $p^*(z)$ is a Boltzmann--like ansatz with coefficients $\beta_a$ related to the distribution’s moments.
            %
            As we will see, this exponential--family form is central to our moment--based unfolding method.
            %
            It provides a flexible yet principled parameterization of the true distribution in terms of a finite set of parameters ${\lambda_a}$ or equivalently a finite set of moments.
            %
            Crucially, if the list of moment constraints is extended and refined, $p^*(z)$ can approximate the true distribution arbitrarily well (approaching the actual $p_Z(z)$ as the number of moments grows).
            %
            Hence the full differential cross section can be reached in the limit of sufficiently many moment constraints.

    \subsection{Unfolding with Moment Constraints: Classical and Modern Approaches}
        Many unfolding methods, both classical and modern, can be interpreted as using moment constraints or related regularization assumptions to tackle the ill--posed inversion of detector effects.
        %
        In a binned setting, each bin count can be seen as a moment (an integral of the continuous distribution against a top--hat basis function).
        %
        Unfolding those bin counts with minimal noise amplification often requires additional constraints such as limiting the number of iterations.
        %
        Although not usually thought of in this fashion, this can be equivalently reformulated as effectively limiting the space of possible moment values of these distributions convolved with a top--hat function.
        
        \subsubsection{Linear Regularized Unfolding (Tikhonov and SVD)}
            One class of unfolding methods formulates the problem as a linear system $mu_i = \sum_j R_{ij} \nu_j$, where $mu_i$ are observed counts in detector bin $i$ and $\nu_j$ are the true distribution values (e.g. cross section in true bin $j$).
            %
            Solving for $\nu_j$ directly (e.g. matrix inversion or unregularized maximum likelihood) is notoriously unstable.
            %
            Although we have discussed this in \cref{chap:theoretical-foundations}, we are now equipped to reformulate the problem in the language of moments.
            %
            High--frequency fluctuations in $\nu$, equivalent to variations in high--order “moments”, can fit statistical noise in $\mu$.
            %
            Tikhonov regularization addresses this by adding a penalty on undesirable solutions, usually favouring smoothness.\kd{A.N. Tikhonov, On the solution of improperly posed problems and the method of regularization,
Sov. Math. 5 (1963) 1035.}
            %
            For example, one penalizes the squared second derivative of the unfolded spectrum or deviations from a prior guess.
            %
            This effectively constrains the higher--order moments of the solution (suppressing oscillatory components that are poorly determined by data). The result is a bias toward smooth moment behaviour, trading a bit of bias for a dramatic reduction in variance.\kd{https://www.ippp.dur.ac.uk/Workshops/02/statistics/proceedings/cowan.pdf}\kd{see references}
            
            Similarly, the singular value decomposition (SVD) unfolding method truncates small singular values of the response matrix.\kd{Michael Schmelling, The method of reduced cross-entropy. A general approach to unfold probabil-
ity distributions, Nucl. Instrum. Methods A340 (1994) 400.}
            %
            This is equivalent to discarding combinations of moments that cannot be determined well (those along directions of the solution space corresponding to tiny eigenvalues would otherwise blow up with noise).
            %
            By keeping only the dominant modes, essentially the lowest--frequency or largest--size moments, SVD unfolding ensures stability at the cost of not fully utilizing high--frequency information.
            
            Both Tikhonov and SVD thereby regularize the moment space, either explicitly or implicitly limiting the effective number of moments that contribute to the unfolded solution.\kd{Cowan}
        \subsubsection{Iterative Bayesian Unfolding (D’Agostini)}
            The iterative method by D’Agostini \kd{cite}(known by various names; in HEP, Iterative Bayesian Unfolding) approaches unfolding as a successive moment matching procedure.
            %
            It starts with an initial guess for the true distribution (the prior Monte Carlo prediction call `generation', which corresponds to some initial moment estimates) and alternately updates the expectations to better agree with the observed data.
            %
            At each iteration, the method reweights the Monte Carlo events by the ratio of data to simulation in each detector bin (effectively adjusting the candidate true distribution’s moments to reduce the discrepancy in those bins).
            %
            This procedure can be interpreted as ensuring that the predicted detector--level counts (moments of the true distribution under the response) match the observed counts, one step at a time.
            %
            The algorithm converges to a solution that maximizes the likelihood (in the limit of many iterations), but in practice one stops after a finite number of iterations to avoid over-fitting statistical fluctuations.\kd{cite cowan}
            %
            Stopping early or adding a prior is a form of regularization: it limits the effective degrees of freedom (moments) that are fitted, similar to Tikhonov or SVD albeit via a different mechanism.
            %
            This iterative method has the advantage of intuitively incorporating a prior (the initial guess acts as a prior shape, providing stability if data are sparse) and is widely used due to its simplicity and ability to include prior knowledge.
        
        \subsubsection{Entropy-Based Methods}
            Another class of unfolding techniques explicitly incorporates entropy or information criteria to regularize the solution.
            %
            The maximum entropy (MaxEnt) unfolding approach chooses the unfolded distribution that maximizes entropy subject to reproducing the observed detector data (usually via a likelihood term).\kd{cite cowan references}
            %
            In practice this might mean maximizing $S = -\sum_j \mu_j \ln(\mu_j)$ minus a term for agreement with data.
            %
            The solution tends to be as smooth and featureless as possible (high entropy) unless the data significantly demand a structure. Entropy regularization thus penalizes any extraneous moment fluctuations that are not required by the data, biasing toward a flat distribution in the absence of strong evidence for shape.\kd{cite C. Pruneau
Data Analysis Techniques for Physical Scientists}\kd{Marshall:2001ax}
            %
            Schmelling’s method of reduced cross-entropy\kd{cowan [8]} is a related technique, effectively combining likelihood maximization with an entropy prior.
            %
            By treating the deviation from a prior distribution in terms of Kullback--Leibler divergence (relative entropy), one can incorporate prior knowledge while still preferring the least structured solution beyond that prior
            %
            These methods make the connection to moment constraints explicit in that they treat the unfolding problem as one of satisfying certain expectation values (the data constraints) while maximizing uncertainty elsewhere.
            %
            As discussed, this yields an exponential--family solution.
            %
            In fact, the MaxEnt solution can be written in the form $p(z) \propto \exp(-\sum_a \beta_a f_a(z))$ where $f_a(z)$ are the functions defining the data constraints (for example, indicator functions for each detector bin to ensure those predicted counts match the observed).\kd{cowan}
            %
            This is formally identical to the moment--constrained maximum entropy distribution discussed above; the only difference is the nature of the constraints (data--driven constraints rather than actual physical moments of the distribution).
        \subsubsection{Unbinned and Machine Learning--Based Methods}
            With advances in computation, unbinned unfolding techniques have emerged that avoid histogramming data altogether.
            %
            These methods often use machine learning to compare distributions without bins.
            %
            One prominent example is \textsc{OmniFold},\kd{} an unbinned iterative unfolding approach based on modern machine learning classifiers.
            %
            \textsc{OmniFold} uses a classifier (often a neural network) to distinguish weighted simulation from data; the classifier’s output is used to reweight simulation events such that, after reweighting, the simulation is more similar to the data.
            %
            This is done in an iterative fashion (multiple rounds, including both detector--level and particle--level weights) to converge to a set of per--event weights that yield an unfolded distribution matching the observed data.\kd{}
            %
            While \textsc{OmniFold} does not explicitly constrain low--order moments or use an analytic parametrization, it implicitly matches all features of the distribution that the classifier can discern---effectively attempting to equate the full set of moments by the end of the procedure.
            %
            In each iteration, the classifier focuses on the current differences between data and weighted simulation, which are often in some “mode” of the distribution; iterating allows successively finer differences (higher--order moments or localized shape features) to be corrected.
            %
            This is conceptually similar to performing an increasingly detailed moment matching, guided by the ML classifier as a flexible test statistic.
            %
            Other approaches employ adversarial neural networks or optimal transport metrics to directly learn the unfolded distribution in one go, rather than iteratively.\kd{}
            %
            These can be seen as the next step in unbinned unfolding: using a generator model for the true distribution and a discriminator to enforce that the generator’s folded output looks like the data.
            %
            Such setups, inspired by Generative Adversarial Networks (GANs),\kd{} take advantage of the same idea that ultimately all moments need to match for the generated distribution to be indistinguishable from the data.
            %
            I will delve into one such adversarial approach, Reweighting Adversarial Networks (RAN), shortly.
            %
            For now, suffice it to note that modern ML--based methods are moving toward treating unfolding as a fitting problem, using flexible function approximators and powerful statistical distances (e.g. Wasserstein distance\kd{}) to overcome binning and to handle large dimensional data.
            %
            These methods still must address ill--posedness (e.g. through network architecture choices, training tricks, or implicit regularization like early stopping), but they offer a way to directly target the entire set of distributional degrees of freedom rather than pre--selecting a fixed set of basis functions.
    \subsection{Benefits and Challenges of Moment Unfolding}
        Focusing on moments as the quantities to unfold offers several clear benefits.
        %
        First, moments are often directly related to physics predictions.
        %
        Many theoretical models provide predictions for means, variances, or other moment--like observables (especially in QCD where sum rules and scaling laws involve moments of distributions).
        %
        Unfolding at the level of moments thus yields results that can be compared to theory with minimal further processing.\kd{}
        %
        Second, by compressing the data into a few numbers, moment--based unfolding can greatly reduce statistical fluctuations and noise sensitivity.
        %
        A small set of global features (e.g. the first few moments) can usually be determined more precisely than an entire binned spectrum.
        %
        The variance of estimators is lower since we effectively integrate over many events to get each moment.
        %
        This was one motivation for the development of the Moment Unfolding method;
        %
        measuring moments directly can be more precise and less sensitive to binning choices.\kd{}
        %
        Third, moment unfolding can simplify high--dimensional problems.
        %
        In multi--differential measurements (with many kinematic variables), fully binning the data becomes impractical due to sparsity (“curse of dimensionality”).
        %
        But one might still meaningfully measure lower--dimensional moments, e.g. the average of some observable as a function of another variable, avoiding the need to populate multi--dimensional histograms.
        %
        In summary, moment-based approaches concentrate on the most salient features of the distribution, potentially yielding robust, computationally efficient unfolding results when only those features are of interest.

        However, there are important limitations and challenges inherent to moment--based unfolding.
        %
        By construction, focusing on a limited set of moments discards information contained in the distribution beyond those moments.
        %
        Two very different underlying distributions can share the same few moments.
        %
        Thus unfolding only those moments provides an incomplete picture.
        %
        This non--uniqueness means that moment--based results must be interpreted carefully.
        %
        They answer only the questions they explicitly ask.
        %
        For instance, if only the first two moments (mean and variance) are unfolded, any differences residing in the non--Gaussian shape (skewness, tails, multimodal structure) will go undetected.
        %
        Moment unfolding shifts the ill--posedness to the choice of moments.
        %
        One must assume or hope that the chosen moment set is sufficient to capture the physically relevant differences.
        %
        If an unexpected feature lies outside this span, it will be missed.
        %
        This is connected to the concept of model dependence – choosing an insufficient set of moments imposes a bias (a kind of model) on the unfolding result.

        A few different options exist to regularize the problem.
        %
        For example, if unfolding a large number of moments, one might impose a smooth fal--off in the moment values (since very high--order moments tend to be increasingly sensitive to rare tails).
        %
        In our adversarial Moment Unfolding approach, the number of moments $n$ we choose to unfold acts as a regularization knob: a small $n$ strongly regularizes (since it ignores any structure beyond the $n$th moment), while a larger $n$ allows more detailed structure at the cost of higher variance and potential instability.\kd{}
        %
        This is analogous to the bias–-variance trade--off in classic unfolding: using fewer moments yields a biased but low--variance estimate; using more approaches an unbiased full result but with higher variance (and risk of over--fitting data fluctuations).
        %
        One limitation of any reweighting-based unfolding (moment-based or otherwise) is the requirement of support overlap.
        %
        The method can only correct distributions within the support of the MC (prior) distribution.
        %
        If the true distribution has events in regions where the simulation has zero or negligible events, no amount of reweighting or moment adjustment can cover that gap.
        %
        One would need to generate new events (a different approach entirely) or rely on extrapolation.
        %
        Thus, moment unfolding, like other reweighting methods (\textsc{OmniFold}, RAN, etc.), assumes that the generation's phase space is broad enough to contain the truth (or that any deficiencies are corrected by separate means).
        %
        This is usually ensured by using a sufficiently generic simulation or augmenting it with additional samples if needed.
        %
        It is worth noting that as we increase the number of moments or attempt to unfold the full distribution, the demand on simulation support becomes stricter: essentially all features of the data distribution must be present in the starting sample to be recovered by reweighting.
    \subsection{Extending Moment Unfolding}
        This Ansatz can be viewed as a minimal deformation of the simulation needed to match data.
        %
        $\beta_a = 0$ for all $a$ corresponds to $g(z)=1$ (no reweighting, i.e. using the raw generation as is), and as we turn on $\beta_a$’s, we gently adjust the weight of each MC event based on its $z$ value.
        %
        The exponential form guarantees that no arbitrary structure is introduced beyond that needed to fulfil the moment constraints i.e. it is the maximum entropy solution consistent with those constraints.\kd{}
        %
        Another perspective is to see $g(z)$ as the Radon-–Nikodym derivative between the true distribution $p(z)$ and the generation $q(z)$, restricted to the family of functions parametrized by $\beta_a$.\kd{}
        %
        If the true distribution $p(z)$ lies within this family (for the chosen $T_a$ set), then there exists some $\boldsymbol{\beta}{\text{true}}$ such that $p(z) = g(z; \vb*{\beta}_{\text{true}}),q(z)$.
        %
        Even if $p(z)$ is outside this family, we expect that with a sufficiently rich set of basis functions $T_a$, one can approximate $p(z)$ in the moment sense.
        %
        In summary, Eq.~\ref{eq:moment-reweight} translates our physics question (“what are the true moments?”) into a set of parameters $\beta_a$ to be determined.

        Once the optimal $\beta_a$ parameters are found, the unfolded moments are obtained immediately by computing the weighted averages in the reweighted particle--level sample.
        %
        The unfolded $k$th moment of $Z$ is simply
        \[
            \langle Z^k \rangle_{\text{unfolded}} \;=\; \frac{\int z^k\,g(z)\,q(z)\,\dd z}{\int g(z)\,q(z)\,\dd z} ~,
        \]
        or in practice, the ratio of weighted sums over all generation events.
        %
        By construction, these unfolded moments will agree with the true moments of $p(z)$ if the procedure succeeds in finding $\beta_a$ such that the reweighted simulation matches the data atdetector-level.
        %
        Incorporating the normalization factor $P(\boldsymbol{\beta})$ (the denominator above) is important.
        %
        It ensures that $g(z)$ does not arbitrarily change the total number of events.
        %
        In statistical mechanics language, $P(\boldsymbol{\beta})$ is the partition function ensuring probability conservation.
        %
        In the unfolding context, it guarantees that the reweighted distribution properly normalizes to the total cross section (or total event count) expected.
        %
        There is no explicit formula linking $\beta_a$ to a simple goodness--of--fit measure at detector level, because the detector response $r(x|z)$ is typically a complex stochastic mapping.
        %
        Instead, a machine learning approach evaluates and drives this optimization.
        
        The Moment Unfolding method was demonstrated to yield accurate and statistically stable moment estimates in realistic scenarios.
        %
        By avoiding any intermediate binning, it circumvents bin-size biases and fully utilizes the fine--grained shape information in the data.
        %
        Moreover, it is remarkably efficient.
        %
        Focusing on a handful of parameters $\beta_a$ is much simpler than attempting to learn a full function (as a generative model would).
        %
        In essence, it reduces the unfolding task to a parameter fitting problem under the hood, albeit a high--dimensional, simulation--informed, adversarially--solved one. This focus provided a built--in regularization (as we emphasized, choosing small $n$ limits complexity).
        %
        The results in Section~\kd{sec:results} showed that even with $n$ as small as \(2\) one can capture key shape characteristics of distributions.
        %
        However, ultimately one may wish to unfold the entire distribution without having to pre--select moments.
        %
        This is where we transition to extending the Moment Unfolding concept to all moments, which leads us to the Reweighting Adversarial Networks (RAN) framework.

        While unfolding a fixed set of moments is valuable, the ultimate goal in many analyses is to recover the complete differential cross section, effectively, to determine the true distribution $p(z)$ itself (within resolution limits).
        %
        The Moment Unfolding framework provides a natural stepping stone to this goal. Conceptually, one can imagine increasing the number of moments $n$ in the weight function Eq.~\ref{eq:moment-reweight} to capture finer and finer detail of the distribution.
        %
        In the limit $n \to \infty$ (or including an appropriate functional basis for $T_a(z)$), the weight function $g(z)$ could represent an arbitrarily complex reweighting, capable of morphing the simulation into the true distribution.
        %
        In practice, however, trying to include a very large number of moment parameters directly poses challenges.
        %
        The normalization factor $P(\boldsymbol{\beta})$ (partition function) becomes increasingly complicated to compute or differentiate when $\boldsymbol{\beta}$ is high--dimensional, and the adversarial training may become unstable or inefficient when there are so many degrees of freedom to adjust.
        %
        In the initial Moment Unfolding studies, keeping $n$ small was important for stable training---it acted as a strong regularizer and simplified the learning problem.\kd{}
        
        To extend the method toward full distributions, the Reweighting Adversarial Networks (RAN) approach was developed.\kd{}
        %
        RAN can be seen as Moment Unfolding taken to the continuum limit, where instead of a few predetermined moments, all aspects of the distribution are learned. Practically, RAN forgoes the explicit parametrization of $g(z)$ with a fixed set of $\beta_a$ multiplying known basis functions.
        %
        Instead, it employs a more general function approximator (such as a neural network or a high-capacity ansatz) to represent $g(z)$, which can adjust weights flexibly for different regions of phase space.
        %
        The adversarial setup remains, but now the generator’s parameters are not directly interpretable as specific moments;
        %
        they are a more granular description of the weight function.
        %
        The discriminator in RAN is tasked with comparing the fully continuous shapes of $\tilde{q}(x)$ and $p(x)$, effectively enforcing an infinite number of constraints (since matching two distributions means matching an infinite set of moments or test statistics). In this way, RAN builds on the moment-based approach by removing the artificial limit on the number of moments: the method `learns to unfold all the moments of a distribution'. \kd{}

        One important innovation in RAN was the use of the Wasserstein GAN framework,\kd{} which provides a stable way to train the adversarial system even when the generator distribution and true distribution initially differ significantly.
        %
        The Wasserstein distance (or Earth Mover’s distance) offers a continuous and meaningful loss metric that correlates with distribution differences, which helped guide the training when $g(z)$ has high flexibility.
        %
        This addresses a subtle issue.
        %
        When extending to complete distributions, one must contend with the fact that $g(z)$ estimates the partition function from batch data.
        %
        In a naive GAN, if $g(z)$ significantly changes normalization, the discriminator could easily detect a total rate difference, and the generator would then simply rescale everything to match the event counts, potentially neglecting shape differences.
        %
        By using an optimal transport loss and carefully constraining $g(z)$ to keep the total weight near unity, RAN avoids trivial solutions and focuses on shape adjustments.
        %
        In effect, RAN had to incorporate the partition function normalization into the training.
        %
        Although this was also true of Moment Unfolding, there is a quantitive difference.
        %
        When $n$ was small, a few $\beta$ primarily tweak shape within a mostly normalized scheme, but misestimations of the partition function become very relevant when $g(z)$ becomes a free-form function.\kd{}
        %
        Section~\kd{sec:infiniteunfolding} is devoted to a detailed description of the RAN methodology; this section emphasizes the conceptual link: RAN is the natural extension of Moment Unfolding to the case where the number of moment--like features is unlimited.

        By allowing a more expressive reweighting function, RAN is able to unfold the entire distribution in an unbinned, non--iterative manner.
        %
        This stands in contrast to \textsc{OmniFold}, which, while also ultimately recovering the full distribution, does so via multiple iterative reweighting steps.
        %
        RAN achieves a similar end point in one training loop by leveraging the powerful GAN paradigm.
        %
        However, this power comes with technical challenges: the loss landscape is more complex and the risk of overfitting noise is higher when we essentially have as many ``parameters" as there are events.
        %
        Our implementation of RAN thus required additional care in training (e.g. gradient penalty terms for WGAN, spectral normalization, etc.) to ensure convergence to a physically sensible solution (one that generalizes and doesn’t chase statistical fluctuations).
        %
        The success of RAN in toy examples and real data applications demonstrates that it is indeed possible to go from moments to complete spectra.,

In summary, the progression from Moment Unfolding to full distribution unfolding can be seen as a continuum.
%
At one end, we have a highly regularized method focusing on a few moments (stable but incomplete); at the other end, we have methods like RAN that aim to extract the entire distribution (complete but needing stronger techniques to control fluctuations).
%
Moment Unfolding paves the way to the more ambitious goal of unbinned differential cross section measurement via its extension to RAN.
%
This connection between moments and spectra ensures that the insights and constraints from one domain (e.g. theoretical expectations for certain moments) can be seamlessly integrated into the process of obtaining the full spectrum.
%
By casting the unfolding problem in the language of moment constraints and exponential families, one can gain both intuitive and quantitative control over the unfolding, whether one stops at a few moments or pushes onward to recover the complete distribution.
%
This framework not only bridges classical and modern techniques but also provides a clear rationale for why unfolding even a handful of moments is a stepping stone to unfolding everything: the moments are the fingerprints of the distribution, and once one learns to reliably recover those fingerprints (even individually), one is well--equipped to tackle the entire handprint that is the full differential cross section.

\section{Methodology and Regularization}
    A variety of regularization strategies have been developed historically and in contemporary machine learning (ML) approaches to ensure stable, physically meaningful unfolding solutions.
    %
    This section begins by reviewing these strategies, from classical techniques to modern ML--based methods, before detailing how the Reweighting Adversarial Networks (RAN) methodology is designed to address regularization.
    %
    The focus will be on the theoretical and conceptual aspects of the RAN approach, rather than low--level implementation details.

    \subsection{Historical Approaches to Regularization}
        Classical unfolding methods introduce explicit regularization to tame the ill--posed nature of the problem.
        %
        A prominent example is {Tikhonov regularization}, which adds a penalty on the curvature or norm of the solution to the fitting objective.\kd{}
        %
        In a typical binned unfolding scenario with response matrix $\mathbf{R}$ relating true binned spectrum $\boldsymbol{\nu}$ to measured data $\boldsymbol{\mu}$, Tikhonov regularization finds the unfolded estimate by minimizing a modified $\chi^2$:
        \begin{equation}
            \hat{\boldsymbol{\nu}} = \underset{\vb*\nu}{\arg\min} \Big[\,|\mathbf{R}\boldsymbol{\nu} - \boldsymbol{\mu}|^2 + \lambda\,| \mathbf{L}(\boldsymbol{\nu}-\boldsymbol{\nu}_0)|^2 \Big]~,
        \end{equation}
        where $\mathbf{L}$ is typically a discrete derivative (smoothness) operator and $\boldsymbol{\nu}_0$ is a prior estimate.\kd{}
        %
        The regularization parameter $\lambda$ controls the bias–variance trade-off\kd{}.
        %
        $\lambda\to 0$ recovers an unbiased but high--variance solution, while large $\lambda$ biases $\hat{\boldsymbol{\nu}}$.
        %
        Techniques such as L--curve scans or cross--validation are used to choose $\lambda$ optimally.\kd{}
        %
        Tikhonov--style regularization (implemented for instance in \textsc{TUnfold}\kd{}) is widely used in precision measurements for its ability to suppress statistical noise when the true distribution is expected to be reasonably smooth\kd{cite}.
        %
        Related linear regularization schemes include {singular value decomposition (SVD) truncation}, which diagonalizes the response matrix and discards or down-weights contributions along directions with small singular values (which are dominated by noise)\kd{cite}.
        %
        This effectively regularizes the solution similarly to a Tikhonov cutoff, by eliminating high-frequency oscillatory components that are weakly constrained by data\kd{cite}. 
        %
        Another classical strategy is {iterative Bayesian unfolding} (often called D’Agostini or Richardson-–Lucy iterative method)\kd{cite}.
        %
        In this approach, an initial guess (prior) for the truth distribution is repeatedly updated using Bayes’ theorem and the observed data.
        %
        The regularization is introduced by not iterating to full convergence; stopping after a finite number of iterations (usually a few) serves as an implicit regularization that prevents overfitting to statistical fluctuations.\kd{}
        %
        The number of iterations thus plays a role analogous to $\lambda$ in controlling the smoothness of the outcome, with early stopping yielding a more biased (closer to the prior) but stabler result.\kd{}
        %
        Finally, {maximum entropy} methods introduce an entropic prior, favouring solutions that maximize information entropy subject to data constraints\kd{cite}. 
        %
        By penalizing deviations from a featureless (flat or known prior) distribution, maximum-entropy unfolding produces a smooth unfolded spectrum unless the data strongly indicate otherwise.
        %
        These “semi-classical” methods (e.g. SVD truncation, MaxEnt) were developed to reduce ad--hoc tuning.
        %
        For instance, maximizing entropy automatically regularizes by pushing the solution toward the least--informative shape consistent with the measurements\kd{cite}.
        %
        All of these classical techniques acknowledge the need for external constraints to obtain a stable inversion of the detector response.

        In recent years, unbinned and high-dimensional unfolding methods based on machine learning have emerged, necessitating new ways to incorporate regularization.\kd{}
        %
        Many ML-based approaches avoid explicit binning (thus eliminating bin--size artifacts) but must still combat the same amplification of statistical noise.
        %
        Often, the regularization in ML approaches is \emph{implicit}, coming from network architectures and training protocols.\kd{}
        %
        For example, deep neural networks have a finite capacity and tend to learn simpler patterns first, a form of Occam’s razor that can act as a prior.
        %
        Choices of network depth, width, and activation functions impose smoothness biases on the learned mapping.\kd{}
        %
        In convolutional networks, weight sharing and locality encode translational invariance, effectively constraining the space of possible unfoldings to those that respect certain symmetries.\kd{}
        %
        Many ML unfolding methods also apply standard regularization techniques from predictive modelling, such as weight decay ($L^2$ penalties)\kd{cite}, dropout\kd{cite}, or batch normalization, to prevent overfitting.
        %
        These generic measures help ensure the learned model does not simply reproduce statistical fluctuations in the training sample.

        In adaptive or iterative ML unfolding algorithms, an explicit analogue of early stopping is used.
        %
        The \textsc{OmniFold} method, for instance, performs successive reweighting iterations with neural classifiers.\kd{}  By limiting the number of iterations (often on the order of \(4-6\)) and monitoring when the updates become consistent with statistical uncertainty, \textsc{OmniFold} effectively regularizes the solution.\kd{}
        %
        If iterated to completion, an algorithm like \textsc{OmniFold} would overfit the data (analogous to iterative Bayesian unfolding); halting earlier mitigates this risk.\kd{}
        %
        Neural network--based unfolding methods also permit incorporation of physics--informed constraints as a form of regularization.
        %
        For example, one can modify the loss function to penalize unphysical deviations or enforce known conservation laws and symmetries\kd{cite}.
        %
        Recent studies have integrated exact symmetry constraints into normalizing flow or invertible network architectures used for unfolding, so that the learned particle--level distribution automatically preserves quantities like momentum or charge\kd{cite}.
        %
        This reduces the solution space a priori to physically plausible ones, which is a powerful regularization aligning with domain knowledge.

        Beyond reweighting approaches, generative models learn to generate events from scratch so that, when passed through a detector simulation, they reproduce the observed data.\kd{}
        %
        Examples include normalizing flows and variational autoencoders trained to invert detector effects\kd{cite}.
        %
        These models face even greater risk of instability due to their high flexibility.
        %
        Regularization here can take the form of strong prior distributions (for instance, seeding a flow with a known prior and only allowing limited deviation)\kd{cite}, or carefully tuned network constraints.
        %
        In practice, the success of ML generative unfolding has been limited by the need for large training samples and the difficulty of covering the full phase space without introducing spurious modes\kd{cite}.
        %
        Indeed, most existing ML unfolding methods (\textsc{OmniFold} and variants\kd{cite}, flow-based methods\kd{cite}, etc.) still assume that the simulated data sample has sufficient overlap with the true distribution in all regions of interest.\kd{}
        %
        If the simulation has zero or very little support in some region that the real data populate, no amount of reweighting or network fitting can compensate; this is the well-known curse of extrapolation in unfolding.

    \subsection{Regularization in RAN}
    \label{subsec:regularisation-in-ran}
        The Reweighting Adversarial Network approach is expressly designed to address regularization challenges while performing unbinned unfolding in one stage.
        %
        RAN can be viewed conceptually as an extension of the \emph{Moment Unfolding} method\kd{cite}, which unfolded only a fixed set of distribution moments and thereby enjoyed a strong built--in regularization (by restricting to a few summary statistics).
        %
        RAN lifts that restriction by attempting to unfold the entire distribution (the “infinite--moment” limit), requiring new regularization techniques to keep the problem well--defined.\kd{}
        %
        Unlike \textsc{OmniFold}, which iteratively trains many separate classifiers, RAN solves for all weights in a single training round by jointly optimizing $g$ and $d$.
        %
        This provides a significant computational advantage, but it demands a very stable training procedure because the full flexibility of $g(z)$ is unleashed at once.

        To ensure stable and meaningful solutions, RAN incorporates several regularization driven design choices in its methodology.
        %
        \subsubsection{Wasserstein GAN}
            First and foremost, RAN employs a {Wasserstein GAN (WGAN)} framework for the adversarial training, in place of a standard (Jensen–Shannon divergence-based) GAN.
            %
            In a vanilla GAN, the discriminator $d(x)$ is trained to output a binary classification (real vs fake) and the generator is trained to fool it, typically using a binary cross--entropy or similar loss.
            %
            Such losses correspond to minimizing a divergence like Kullback-–Leibler or Jensen–-Shannon; if the real and model distributions do not overlap, the gradient of these losses vanishes, leading to unstable or mode--collapsed solutions.
            %
            By contrast, RAN leverages the Earth Mover (Wasserstein-1) distance as the measure of difference between distributions.\kd{}
            %
            The WGAN critic $d(x)$ does not output a class label but rather a real--valued score, which should be high for real data and low for simulated (reweighted) data.\kd{}
            %
            The generator seeks to maximize this critic score on the simulated data, thereby minimizing the Wasserstein distance between the weighted simulation and the true data distribution.
            %
            The objective function adopted in RAN’s training can be written as
            \begin{align}
                \label{eq:ran_wgan_loss}
                L_{\text{WGAN}}[g,d] &= \mathbb{E}_{x\sim \text{Data}}\big[d(x)\big] - \mathbb{E}_{(z,x)\sim \text{(Gen., Sim.)}}\big[g(z)\,d(x)\big]\\
                \nonumber&+ \lambda\,\mathbb{E}\!\Big[(|\nabla_{\hat{x}}d(\hat{x})| - 1)^2\Big]~,
            \end{align}
            where $x$ denotes a detector-level event from the real data or simulation, and $(z,x)\sim\text{(Gen., Sim.)}$ indicates a MC event with truth features $z$ and corresponding detector--level observation $x$ (the MC is generative, so $z\to x$ via the detector model).
            %
            The first two terms estimate the Wasserstein-1 distance between the true and reweighted simulated distributions.
            %
            The last term is a \emph{gradient penalty}\kd{cite} (with strength $\lambda$) enforcing the requirement that $d(x)$ be \(1-\)Lipschitz continuous, as required for the Wasserstein distance theory.\kd{}
            %
            This penalty drives $|\nabla d|$ toward unity on random points $\hat{x}$ interpolated between real and simulated samples, a technique introduced in Ref.~\kd{cite} to stabilize WGAN training.
            %
            The minimax game on the loss \eqref{eq:ran_wgan_loss} with respect to $g$ and $d$ corresponds to the following adversarial logic:
            %
            \begin{enumerate}
                \item $d$ tries to minimize $L_{\text{WGAN}}$ (making the data–sim discrepancy as large as possible by assigning higher scores to real data than to any weighted sim sample)
                \item $g$ tries to maximize it (reducing discrepancy by adjusting weights $g(z)$ to make the sim look as “real” as possible).
            \end{enumerate}
            At equilibrium, the weighted simulation cannot be distinguished from data by any \(1-\)Lipschitz function, meaning the distributions are empirically matched in the optimal transport sense.
    
            The choice of a WGAN loss is fundamentally motivated by regularization considerations.
            %
            Because the Wasserstein distance evaluates the “distance” between distributions in terms of an optimal transport cost, it provides smooth, meaningful gradients even when the support of the model and data distributions do not overlap perfectly.\kd{}
            %
            In other words, if the simulation is deficient in some region of phase space, a standard GAN discriminator would drive $g(z)$ to extreme values (or simply saturate, providing no gradient) in an attempt to cover the deficit, often leading to unstable oscillations in $g(z)$ (very large weights assigned to a few events).
            %
            The WGAN critic, however, will assign a large positive $d(x)$ to real data regions with no simulated equivalent, but the gradient of its linear output will encourage nearby simulated events to increase in weight in a more controlled fashion, effectively telling the generator \emph{which direction} to move probability mass to reduce the discrepancy.
            %
            This greatly mitigates mode collapse and high-variance weight solutions.\kd{}
            %
            In our context of an “infinite--moment” unfolding (unconstrained by moment selection), using the WGAN framework is a key to avoiding the generator thrashing that occurred with a naive GAN approach.
            %
            Empirically, we find that replacing the binary cross--entropy GAN loss with the WGAN loss stabilizes the training and yields a smoother weight distribution $g(z)$, especially in sparse regions of the detector feature space.
            %
            An additional benefit is that RAN can handle cases with {minimal detector-level overlap} between simulation and data.
            %
            Since the Wasserstein metric remains finite as long as the support of the weighted simulation can be transported to cover data (even if initially disjoint), RAN does not strictly require the simulation and data distributions to overlap bin--by--bin.
            %
            Of course, overlap at particle level is still required.
            %
            RAN cannot invent truth--level events outside the generation’s support.
            %
            Nonetheless, the ability to tolerate detector--level mismatches is a major advantage over methods that rely on probability ratios (which diverge when supports do not overlap).
            %
            In essence, the use of WGAN in RAN provides a strong form of regularization--by--design: it selects a smoother notion of distribution difference that avoids infinite gradients and excessive weight variance.
        \subsubsection{Regularizing the critic: Lipschitz Constraints}
            The critic network $d(x)$ in RAN is subject to the \(1-\)Lipschitz constraint required by the Wasserstein theory.
            %
            We enforce this through two complementary regularization techniques.
            %
            The first, already noted, is the \textbf{gradient penalty} term in Eq.~\eqref{eq:ran_wgan_loss}\kd{cite}.
            %
            Rather than clipping weights (the original WGAN approach\kd{cite}, which can impede learning by reducing capacity), we adopt the improved strategy of penalizing the norm of the critic’s gradient on random interpolations to be close to 1.\kd{}
            %
            This encourages $d(x)$ to remain within the space of \(1-\)Lipschitz functions without harsh constraints on its parameters.
            %
            The second technique is \textbf{spectral normalization} applied to the critic’s layers.\kd{}
            %
            Spectral normalization rescales the weight matrix of each layer such that its largest singular value is 1\kd{cite}.
            %
            By doing so at every training step, we effectively bound the Lipschitz constant of each linear component of $d(x)$, ensuring that $d(x)$ cannot change faster than a certain rate with respect to its input.\kd{}
            %
            Even small violations of the Lipschitz condition can lead to training instabilities (the critic might exploit them to push $L_{\text{WGAN}}$ lower, causing $g$ to react with wildly large weights).
            %
            Spectral normalization prevents the critic from assigning arbitrarily large scores to individual samples.\kd{}
            %
            This has an important regularizing effect on the generator’s behaviour.
            %
            If $d(x)$ is bounded and smooth, the incentives for $g(z)$ to produce extremely large weight factors are reduced, because no single event can ever receive a disproportionately large score advantage.
            %
            In effect, spectral normalization caps the influence of any one data–-simulation discrepancy on the loss.
            %
            The combined use of gradient penalties and spectral normalization in RAN was found to be effective for training stability---they keep the critic in check so that the adversarial game remains in a regime where gradients are informative and the generator’s updates remain moderate.
            %
            With these in place, we substantially avoid the divergence and mode--collapse issues that plague naive adversarial training.\kd{}

        \subsubsection{Regularizing the Generator: Initialization and Activation Constraints}
            On the generator side (the reweighting function $g(z)$), RAN incorporates two important measures to regularize the solution.  First, we initialize $g(z)$ to close to the identity function before beginning adversarial training.
            %
            In the context of unfolding, the “identity” reweighting corresponds to assigning weight 1 to every MC event (i.e. initially assuming the Monte Carlo truth distribution is correct and no reweighting is needed).
            %
            We realize this by initializing the neural network representing $g(z)$ such that its output $\approx 1$ for all $z$.
            %
            This initialization reflects our prior belief that the generator only needs to make relatively small, smooth adjustments to the starting simulation, which is a reasonable assumption in many cases where the Monte Carlo is tuned to approximate reality.
            %
            By starting close to the identity mapping, we avoid a situation where early in training the critic finds large differences and drives $g$ to extreme compensations.
            %
            Large fluctuations in weights early on can kick off a feedback loop of instability (the critic then sees a very irregular simulated distribution and reacts pathologically, etc.).
            %
            The initialization acts as a strong regularizer by biasing $g$ toward the null reweighting unless the data indicate otherwise.
            %
            This idea is analogous to using a Bayesian prior equal to the simulation and regularizing toward it at the start; only genuine discrepancies will pull $g$ away from \(1\).
            %
            Indeed, from the perspective of classical regularization, our weight initialization is equivalent to choosing the simulated distribution as a prior and initially penalizing deviations from it.
            %
            We find that this leads to a much smoother evolution of the adversarial game---initially the critic cannot easily tell data from reweighted simulation because $g\approx 1$ still leaves them broadly similar, so $d$ learns gradually to distinguish the two, and $g$ then adapts gradually in turn.
            %
            This procedure significantly reduces the risk of the training falling into bad local minima or diverging in the first epochs.
            %
            In summary, initializing close to the identity provides a “cold start” regularization that keeps RAN in the perturbative regime where it performs best, rather than having to learn an unpredictable transformation from scratch.\kd{}

            The second generator--side regularization is controling the asymptotic behaviour of the {generator output parameterization} for the weights.
            %
            It is essential that $g(z)$ produce positive weights (events can only be reweighted by positive factors) and have the capacity to yield a wide range of values (some events might need weight $\gg 1$ if data are more abundant there, others weight $<1$ if data are scarce). 
            %
            A naive choice is to define $g(z) = \exp[f_\theta(z)]$ where $f_\theta(z)$ is the output of a neural network with no constraints, and the exponential ensures positivity and unboundedness.
            %
            This choice is in a sense very well motivated by the Boltzmann distribution that inspired Moment Unfolding.
            %
            In practice however, one finds that this choice, while mathematically valid, leads to numerical instability.
            %
            The exponential is a very rapidly growing function, so any relatively large output of the network $f_\theta(z)$ would translate into an astronomically large weight, drastically skewing the training.
            %
            Moreover, the gradient of $\exp[f_\theta(z)]$ is proportional to the output itself, so once a weight becomes huge, its gradient is huge too, often causing large and unstable oscillations.
            %
            To remedy this, a custom activation was designed, that \emph{tames} the asymptotic growth of $g(z)$ at large network outputs while preserving the desired mathematical properties.
            %
            In the final layer of the $g(z)$ network, we replace the $\exp$ with a composite function $\log(1 + \mathrm{softplus}(x))$.
            %
            $\mathrm{softplus}(x) = \log(1+e^x)$ is a smooth approximation to a ReLU (it grows linearly for large $x$, but is smooth everywhere)\kd{cite}.
            %
            Wrapping it in $\log(1+\cdot)$ further slows the growth for large inputs: as $x \to \infty$, $\mathrm{softplus}(x)\approx x$, so $\log(1+\mathrm{softplus}(x)) \approx \log(1+x) \sim \log x$.
            %
            Thus for very large raw network output $x$, this activation grows only logarithmically, instead of exponentially or even linearly.
            %
            This adjusted activation guarantees that
            \begin{enumerate}
                \item $g(z) > 0$ for any input (by construction of softplus and log),
                \item \(g(z)\) can still produce arbitrarily large values in principle (no finite upper bound, unlike a sigmoid which saturates) so we do not limit the solution space unduly, but
                \item Extremely large weights are disfavored because pushing $f_\theta(z)$ to $\gg 0$ yields only diminishing returns in $g(z)$.
            \end{enumerate}
            In practice, this means the network would have to invest a lot of capacity to achieve a huge weight on a single event, which is only worth it if that truly reduces the Wasserstein distance significantly.
            %
            Typically, it will be more effective to increase weights more evenly on a group of events covering a region of phase space than to make one weight enormous.
            %
            This activation function thus serves as an \emph{internal regularizer}, curbing the tendency of the solution to form spikes or outlier weights.
            %
            It is important to emphasize that this is a soft constraint.
            %
            Large weights are not forbidden, but the system must pay a price to realize them, much like a physical prior that discourages sharp discontinuities in the solution unless the data demand them.
            %
            The use of this modified activation proves to be crucial for achieving stable training in RAN; with the standard exponential, one observes frequent instances of a single event’s weight blowing up and derailing the fit, whereas the log--softplus activation yields more balanced weight distributions.
            %
        \subsection{Regularizing via MC Prior}
            Finally, it is worth noting an intrinsic form of regularization in RAN that comes “for free”: by construction, the unfolded result is expressed as a reweighted version of the initial generated truth sample.
            %
            This means the unfolded distribution cannot introduce structures that were not present (even in latent form) in the MC.
            %
            In effect, the MC provides a prior support and shape for the solution.
            %
            If the true underlying distribution has features outside the MC’s support, RAN (like any reweighting method) cannot recover them.
            %
            But conversely, it will not produce spurious artifacts that violate known physics encoded in the simulation.
            %
            This is similar in spirit to a Bayesian prior or a template fit: one starts with a template (the generation) and only deform it as necessary to fit the data.
            %
            The closer the generation is to reality, the less adjustment is needed (and the smaller the risk of overfitting).
            %
            RAN’s whole design of “tweaking a known density rather than learning a new one from scratch” is motivated by the desire to leverage this built--in regularization.
            %
            This allows RAN’s solutions to be smoother and more statistically robust than those from unconstrained generative models, precisely because $g(z)$ operates within the scaffolding of the Monte Carlo sample.
            %
            This comes at the cost of some bias if the Monte Carlo is poor (no method can escape that without additional input), but it is a conscious regularization choice favouring stability over unrestricted flexibility.

    In summary, the RAN methodology interweaves modern ML techniques with classical regularization principles to achieve a stable unfolding.
    %
    The use of the Wasserstein distance (WGAN) provides a gentle, physics--aligned way of comparing distributions that avoids the brittleness of binary classification metrics.
    %
    Imposing Lipschitz continuity on the critic via gradient penalties and spectral normalization keeps the adversarial game well--behaved and prevents the discriminator from overemphasizing statistical noise.
    %
    On the generator side, starting from the physical prior (simulation) and limiting the capacity for extreme weights (through initialization and activation choices) anchor the solution close to expected physics and discourage overfitting.
    %
    Together, these innovations allow RAN to unfold full distributions in one shot (non--iteratively) without the severe instabilities that one might fear from an unconstrained GAN approach.
    %
    In effect, RAN attains a balance: it is flexible enough to accurately fit complex, high--dimensional data, yet sufficiently regularized to suppress unphysical oscillations and variance.
    %
    The result is an unfolding method that achieves competitive or superior performance to iterative methods like \textsc{OmniFold}, while operating in a single training pass and maintaining controlled behaviour even in challenging regimes (such as limited detector overlap or low--statistics bins)\kd{cite}.
    %
    The following sections will demonstrate these points quantitatively with examples, but the methodological foundation laid out here is key to understanding why RAN performs as well as it does.
    %
    It marries the strengths of adversarial learning with the hard--earned lessons of regularization from decades of unfolding research, yielding a novel and powerful approach to this classic inverse problem.
\section{ML Implementation}
    \subsection{Neural Network Architecture}
        A Reweighting Adversarial Network (RAN) consists of two components: a \emph{generator} network that assigns event-wise weights, and a \emph{critic} network that evaluates the discrepancy between weighted simulation and data.
        %
        We implemented both as small fully-connected (dense) neural networks using the \textsc{TensorFlow~2/Keras} framework for prototyping \kd{cite}.
        %
        \footnote{The design is straightforward to reproduce in \textsc{PyTorch} using analogous layers and normalization techniques.}
        %
        The architecture was kept intentionally simple to ensure stable training and avoid over--fitting, with identical layer widths for the generator and critic.

        The generator $g(z;\boldsymbol{\beta})$ takes as input the \emph{particle--level} feature vector $z \in \mathbb{R}^{N_T}$ of a generated event (for example, $N_T=1$ for one--dimensional toy data or $N_T=6$ for the multi--observable jet dataset described below).
        %
        It outputs a scalar weight $w=g(z)$ that reweights that event in order to correct the simulation towards data.
        %
        The generator is a feed--forward multilayer perceptron with three hidden layers of 50 nodes each.
        %
        Each hidden layer uses a Leaky Rectified Linear Unit (Leaky ReLU) activation \kd{cite maas2013rectifier}.
        %
        We apply batch normalization and a dropout of 20\% after each dense layer to promote stable training and reduce overfitting.
        %
        The final layer of the generator is a single linear neuron (no activation) producing a raw scalar $t$.
        %
        This output is then transformed to a positive weight.
        %
        Instead of a direct exponential mapping (which can be prone to numerical instability for large positive or negative $t$), we employ a stabilized transformation,
        \[ 
            w \;=\; \ln\!\Big(1 + \text{softplus}(t)\Big)\,.
        \]
         Finally, the set of weights ${w_i}$ in each mini--batch is normalized so that their average is unity.
         %
         This batch--level normalization means the total weight of the simulated sample remains consistent (preserving overall event counts) and prevents trivial solutions where the network could simply scale all weights up or down without improving the relative distribution shape.
         %
         The generator network thus defines a differentiable reweighting function $g(z)$ that can smoothly adjust the contribution of each simulated event during training.

        The critic $d(x;\boldsymbol{\theta})$ is a discriminator--like network that takes as input the {detector--level} feature vector $x \in \mathbb{R}^{N_D}$ of an event and outputs a real--valued score.
        %
        This network is tasked with distinguishing the \textit{weighted} simulated data from the true data in the detector space, and its output is used as an approximate distance measure between the two distributions.
        %
        Like the generator, the critic is implemented as a fully--connected network with three hidden layers of 50 nodes each, using Leaky ReLU activations and interleaved batch normalization and 20\% dropout.
        %
        The output layer is a single linear node producing the critic score $d(x)$.
        %
        To satisfy the requirements of the Wasserstein GAN framework, the critic must be a \(1-\)Lipschitz function.
        %
        We enforce this constraint via {spectral normalization} on each dense layer’s weights \kd{cite}{DBLP:journals/corr/abs-2009-02773}.
        %
        As discussed, spectral normalization scales the layer weights such that their largest singular value is 1, effectively controlling the Lipschitz constant of $d$ without the need for weight clipping.
        %
        This technique, combined with dropout, greatly improved training stability in the unbounded--weight setup by preventing the critic from becoming overly sensitive to outlier events.
        %
        Additionally, we apply a mild output clipping on the critic scores \kd{cite}, capping $d(x)$ within a reasonable range to eliminate spurious large values that could arise from statistical fluctuations (which in our context would correspond to unphysical negative probabilities or overly large separation scores).
        %
        A schematic of the RAN architecture is shown in Fig.~\ref{fig:model-arch} \kd{(fig: model architecture)}.
        %
        The balanced capacity of the generator and critic (each with $\mathcal{O}(10^4)$ trainable parameters) was found sufficient to learn the required reweighting functions without overfitting, given the complexity of the tasks considered.

    \subsection{Adversarial Training Procedure}
        Training a RAN involves a minimax game between the generator and critic, similar in spirit to a Generative Adversarial Network (GAN).
        %
        However, unlike a standard GAN that generates new events, our generator only reweights existing events, and we adopt the Wasserstein GAN (WGAN) approach \kd{cite}{pmlr-v70-arjovsky17a} for improved stability.
        %
        The critic’s objective is to maximize the statistical distance (in the Wasserstein-1 sense) between the weighted simulation and the true data distributions at the detector level, while the generator’s objective is to produce weights that minimize this distance.
        %
        Formally, let $p(x)$ be the true data distribution in detector space and $q(z, x)$ be the joint distribution of generation and simulation.
        %
        At each training step, the critic $d$ is trained to minimize the loss function
        \[
            L_d = -\mathbb{E}_{x\sim p}[\,d(x)\,]\;+\;\mathbb{E}_{(z, x)\sim q}\big[\,g(z)\,d(x)\,\big]\,,
            \label{eq:wgan_loss}
        \]
        where $(z, x)$ denotes corresponding generation and simulation event pairs.
        %
        $-L_d$ is an empirical estimate of the Wasserstein\(-1\) distance between the true distribution and the reweighted simulated distribution.
        %
        The generator, on the other hand, is trained to maximize this same quantity (equivalently, to minimize the negative critic loss $-L_d$), thereby pushing the weighted simulation to resemble the data.
        %
        Unlike the binary cross--entropy loss in a traditional GAN discriminator, the Wasserstein formulation provides a smooth, continuous loss landscape even when the two distributions do not overlap perfectly \kd{cite arjovsky2017wassersteingan}.
        %
        This is helpful for unfolding problems because even if detector effects cause $p(x)$ and the original $q(x)$ to have disjoint support in $x$, the WGAN critic can still provide informative gradients to the generator.
        %
        This choice eliminates the severe mode collapse or large weight oscillations that occur when using the BCE loss in extant classifier based methods (the WGAN’s stronger theoretical guarantees, combined with the property that RAN only classifies at detector--level, and only reweights at particle--level, ensure the training remains well-behaved even with minimal overlap) \kd{cite gulrajani2017improvedtrainingwassersteingans}.
        %
        To further enforce the \(1-\)Lipschitz condition required by the Wasserstein theory, one can include a gradient penalty term \kd{cite gulrajani2017improvedtrainingwassersteingans} in the critic’s loss.
        %
        This penalty,
        \[
            L_{GP} = \lambda\,\mathbb{E}_{\hat{x}}\qty[(\norm{\nabla_{\hat{x}}d(\hat{x})}_2 - 1)^2],
        \]
        is computed on random points $\hat{x}$ interpolated between real and generated samples.
        %
        $\lambda$ usually set between five and ten.
        %
        The gradient penalty, together with spectral normalization, acts as a robust regularizer against critic over--training.

        Training proceeds by alternating between critic and generator updates in each iteration, following the typical WGAN--GP strategy.
        %
        We used an update ratio of 3:2 (critic:generator), meaning the critic is updated slightly more frequently to keep it near optimality as the generator evolves \kd{cite}{arjovsky2017wassersteingan}.
        %
        Pseudocode for one training cycle is outlined in Algorithm~\ref{alg:wgan-gp}.

        \begin{algorithm}
            \caption{WGAN‑GP Training Step}
            \label{alg:wgan-gp}
            \KwIn{
              Data distribution $p(x)$, 
              generation--simulation pairs \(q(z, x)\),
              critic $d_{\theta}$, 
              generator weights $g(z;\beta)$,\\
              gradient‐penalty weight $\lambda$, 
              critic steps $n_c$, 
              generator steps $n_g$, 
              batch size $B$
            }
            \KwOut{Trained parameters $\theta$ and $\beta$}
        \For{iteration $=1,2,\dots$}{
          \tcp{Critic updates}
          \For{\(i\) in \(1\) \KwTo $n_c$}{
            Sample $\{x_i^{\text data}\}_{i=1}^B\sim p(x)$\;
            Sample $\{(z_j, x_j\}_{j=1}^B\sim q(z, x)$\;
            Compute $w_j = g(z_j;\beta)$\;
            Compute
            \[
              L_d
              = \frac1B\sum_{i=1}^B d_{\theta}(x_i^{\text data})
                -\frac1B\sum_{j=1}^B w_j\,d_{\theta}\bigl(x^{\text sim}_j\bigr)
                +\lambda\,L_{\text GP}(\theta)\,;
            \]
            $\theta \;\leftarrow\;\theta + \eta_d\,\nabla_{\theta}L_d$\;    
          }
        
          \tcp{Generator updates}
          \For{\(j\) in \(1\) \KwTo $n_g$}{
            Sample fresh $\{x_i^{\text{data}}\}_{i=1}^B\sim p(x)$\;
            Sample $\{(z_j, x_j\}_{j=1}^B\sim q(z, x)$\;
            Compute $w_j = g(z_j;\beta)$\;
            Compute
            \[
              L_g 
              = -\frac1B\sum_{j=1}^B w_j\,d_{\theta}\bigl(x_j^{\text{ sim}})\bigr)
                + \frac1B\sum_{i=1}^B d_{\theta}(x_i^{\text{data}})\,;
            \]
            $\beta \;\leftarrow\;\beta - \eta_g\,\nabla_{\beta}L_g$\;
          }
        }
        \end{algorithm}
        These steps constitute one training iteration, and they are repeated until convergence criteria are met (typically tens of thousands of mini--batch iterations, depending on dataset size).
        %
        We optimize both networks using the RMSProp optimizer \kd{cite}, which we found to outperform Adam in our WGAN setting (consistent with previous studies \kd{cite DBLP:journals/corr/abs-1810-02525}).
        %
        The learning rate for both generator and critic was set in the range $\eta \sim 1\times10^{-4}$ to $5\times10^{-4}$ (with the exact value tuned per dataset; for the toy example we used the lower end, while the more complex jet data benefited from a slightly higher rate around $2\times10^{-4}$).
        %
        We did not observe the need for explicit learning rate decay schedules beyond those automatically implemented by \textsc{Keras},\kd{} as training typically converged to a stable solution before any signs of stalling.
        %
        A mini-batch size of $B=256$ was used for all experiments, providing a good balance between gradient estimate stability and computational efficiency.
        %
        We note that we did not perform an exhaustive hyperparameter search; instead, we relied on standard GAN/WGAN settings from the literature and minor tuning.
        %
        In practice, we found the performance of RAN to be robust against moderate changes in these settings: for instance, altering the learning rate by a factor of two or using 40 or 60 nodes per layer (instead of 50) had no significant impact on the final unfolded distributions.
        %
        This insensitivity suggests that the RAN method does not require fine--tuned hyperparameters to achieve reliable results, an important consideration for reproducibility.
    
        Throughout training, we continuously monitor key loss terms to ensure the optimization remains stable.
        %
        In particular, we track the Wasserstein critic loss $L_d$ and the gradient penalty magnitude.
        %
        A sharp increase in the gradient penalty or a sudden large oscillation in $L_d$ can indicate that the critic is violating the Lipschitz constraint or that the generator is producing pathological weights (a potential onset of mode collapse).
        %
        If such behaviour is observed, one can intervene by pausing training and lowering the learning rate or increasing regularization.
        %
        In our experiments, the use of spectral normalization and gradient penalty together largely prevented such divergences.
        %
        We emphasize that RAN’s one--shot training (as opposed to an iterative approach) simplifies the overall optimization schedule: once the adversarial equilibrium is reached, we obtain the final reweighting function without needing multiple retraining cycles.
    
    \subsection{Datasets and Preprocessing}
        We applied the above architecture and training procedure to two representative unfolding scenarios: a controlled \(1-\)dimensional Gaussian example and a realistic high--dimensional jet physics example.
        %
        These two cases allow us to validate the RAN implementation on both simple and complex distributions.

        \subsubsection{Gaussian Example}
            The first dataset is a synthetic scenario with known analytical truth, designed to illustrate RAN’s behavior under varying degrees of detector smearing.
            %
            We generate two sets of events from normal distributions at the particle level, a “truth” distribution $Z_{\text{T}} \sim \mathcal{N}(\mu_{\text{true}},,1)$ and a “generator” (simulation) distribution $Z_{\text{G}} \sim \mathcal{N}(\mu_{\text{gen}},,1)$, with means chosen to be $\mu_{\text{true}}=0$ and $\mu_{\text{gen}}=-1$.
            %
            We sample $10^4$ events from the truth distribution and $10^5$ events from the generation distribution.
            %
            To emulate detector effects, each particle-level value $z$ is transformed by a simple deterministic response: $x = S \cdot z$, where $S$ is a scalar “smearing” factor.
            %
            For example, with $S=1$ the detector--level measurement equals the true value (no smearing), while $S>1$ stretches the distribution, reducing the overlap between the “data” ($x = S\cdot z_{\text{T}}$) and “simulated” ($x = S\cdot z_{\text{G}}$) detector--level observations.
            %
            We consider a range of $S$ values to progressively degrade the overlap.
            %
            For this one--dimensional setup, $N_T = N_D = 1$ (each event is described by a single number).
            %
            Apart from the deterministic scaling, no additional noise or inefficiency is introduced, so the challenge lies purely in the shift of distributions and the lack of detector--level support overlap when $S$ is large.
            %
            We do not apply any additional preprocessing to the input features since they are already standardized (mean shifts aside) and bounded.
            %
            During training, at each iteration we draw a random batch of $B=256$ values from the $10^4$ available smeared data points and another batch of 256 from the $10^5$ smeared simulation points.
            %
            Because the simulation sample is larger, we cycle through it (randomly shuffled) such that multiple data epochs correspond to one pass through the simulation.
            %
            It is beneficial to shuffle and reshuffle both samples frequently during training to avoid any periodic artifacts given the fixed deterministic smearing.
            
            The RAN is tasked with learning event weights $g(z)$ that correct the generation distribution (centered at $-1$) to match the true distribution (centered at 0), using only classification information from the $x$ domain where in extreme cases the distributions barely overlap.
            %
            This toy example is especially useful for validating the implementation because the underlying true reweighting function is known (in the limit of infinite statistics, the optimal weight would be proportional to the ratio of target to sim densities at $z$, which in this case is $\exp[-(z-\mu_{\text{true}})^2/2 + (z-\mu_{\text{gen}})^2/2]$, a Gaussian weight). It also allows us to verify that RAN’s adversarial training indeed converges to the correct solution, by comparing the unfolded distribution to the analytic expectation.

        \subsubsection{Jet Substructure Dataset}
            The second dataset studied is a realistic example taken from high--energy particle physics, involving the unfolding of jet substructure observables.
            %
            We follow the setup of the \textsc{OmniFold} study \kd{cite Andreassen:2019cjw} to enable direct comparisons.
            %
            Proton-proton collision events at $\sqrt{s}=14$~TeV were generated using two different Monte Carlo simulators, \textsc{Pythia}8.243 \kd{cite Sjostrand:2014zea,Sjostrand:2007gs} (with CMS Tune26 \kd{cite ATL-PHYS-PUB-2014-021}) provided the initial \emph{generation} sample, and \textsc{Herwig}~7.1.5 \kd{cite Bahr:2008pv,Bellm:2017bvx} was used to generate an independent sample treated as the \emph{truth} target distribution.
            %
            Each event in both samples contains a high-$p_T$ $Z$ boson (with $p_T^Z > 200$~GeV) decaying leptonically, produced back--to--back with a hadronic jet.
            %
            The presence of a $Z$ boson trigger ensures a well--defined event sample and reduces acceptance differences.
            %
            The final--state particles in each event were passed through a fast detector simulation using \textsc{Delphes}~3.4.2 \kd{cite deFavereau:2013fsa} with a CMS detector configuration (including particle--flow reconstruction \kd{cite Mertens:2015kba,CMS:2017yfk}).
            %
            Jets are then reconstructed in both the particle--level and detector--level event records using the anti$-k_T$ clustering algorithm \kd{cite Cacciari:2008gp} (radius parameter $R=0.4$) as implemented in \textsc{FastJet}3.3.2 \kd{cite Cacciari:2011ma}.
            %
            We focus on the hardest jet in each event (the jet that balances the $Z$ boson’s recoil) to define our observables.
            %
            For each jet, we consider six different substructure observables that characterize its internal pattern of particles.
            %
            These include classic measures like the jet mass $m$ and jet breadth or width $w$, as well as more specialized variables such as the \(N-\)subjettiness ratio $\tau_{21}$, the groomed momentum fraction $z_g$ from the Soft Drop algorithm, and a logarithmic version of the dimensionless jet mass $\ln\rho$.
            %
            Precise definitions of these observables follow those in Ref.\kd{cite Andreassen:2019cjw}, and we refer the reader there or to Appendix \kd{ref} for the formulae.
            %
            In total, each jet is described by a feature vector of $N_T = N_D = 6$ numbers (since the same set of observables is defined at particle--level and, after \textsc{Delphes} simulation, at detector--level).
            %
            These six features form the input to the RAN networks.
            %
            It is helpful to whiten (z--score)\kd{} these inputs to minimize finite precision and overflow errors.
            %
            We prepared the data such that the truth sample (\textsc{Herwig}, particle--level) contained on the order of $10^6$ jets, and the simulation sample (\textsc{Pythia}, particle-level) a similar order of magnitude.
            %
            The detector--level representations for each sample are obtained via the same \textsc{Delphes} procedure, yielding paired $(z, x)$ examples for \textsc{Pythia} and independent $x$ examples for \textsc{Herwig}.
            %
            During training, we treat the \textsc{Delphes+Herwig} jets as our “real data” distribution $p(x)$ and the \textsc{Delphes+Pythia} jets as the initial $q(x)$. %
            At each iteration, mini-batches of $B=256$ jets are drawn from each distribution.
            %
            Given the relatively large size of the samples, we did not need to worry about cycling through the data too many times; we simply train for a fixed number of iterations (determined by early stopping as described next).
            %
            We ensured that each jet in the training sample is used many times over the course of training, which is important for the generator to refine the weights on all regions of phase space.
            %
            Notably, because RAN’s generator applies weights at the particle level (prior to detector simulation), it effectively learns to reweight the \textsc{Pythia} particle-level jets such that, after \textsc{Delphes}, their distribution of observables matches that of the \textsc{Delphes}-processed \textsc{Herwig} jets.
            %
            Once training is complete, those learned weights $w = g(z)$ can be applied directly to the \textsc{Pythia} particle--level events to produce an unfolded distribution for each observable, which we can then compare to the true \textsc{Herwig} particle--level distribution.
            %
            We emphasize that in this example the “data” is actually fully simulated (\textsc{Herwig+Delphes}) rather than collider observations;
            %
            this is a common practice in validation studies, allowing us to quantitatively evaluate the accuracy of the unfolding by comparing to known truth.
            %
            The RAN method, of course, does not rely on knowing the truth distribution during training—it only uses the detector--level information, so it is directly applicable to real data once validated.

    \subsection{Training Monitoring and Validation}
        To ensure reliable performance and prevent overtraining, one should employ a rigorous monitoring and validation regimen during RAN training.
        %
        First, one should partitioned each dataset into a training set (used to update network parameters), a smaller validation set (used only for performance monitoring), and a testing set.
        %
        The ratios used were training:validation:testing \(70:15:15\)
        %
        The validation data is not seen by the networks during gradient updates, but one computes the same losses on it to check how well the model generalizes.
        %
        Specifically, after every fixed number of training iterations (for instance, every 100 mini-batch updates), one evaluates the critic loss $L_d$ on the validation batch and tracks its evolution.
        %
        If one observes the validation loss starting to increase while the training loss continues to decrease, this is a clear sign of overfitting i.e. the critic is learning spurious differences that do not generalize, or the generator is over--adjusting to peculiarities of the training sample.
        %
        In such cases, one should invoke an \emph{early stopping} criterion.
        %
        In our implementation, early stopping was triggered if the validation Wasserstein distance (approximated by $-L_f$ on the validation set) had not improved for 10 consecutive evaluation intervals.
        %
        Once triggered, one stops training and rolls back to the generator state that had the best validation performance.
        %
        This strategy ensures that one did not over--train the critic to exploit statistical noise, which would manifest as unstable weights when applied to new data.

        In addition to the WGAN loss, one can track several high--level divergence metrics between the reweighted generation and truth distributions.
        %
        While these metrics are not used to train the model (since truth is not known in real applications), they are extremely useful to diagnose convergence and compare to alternative methods.
        %
        One such metric is the \textbf{Wasserstein-1 distance} itself between the weighted and true distributions.
        %
        Although our critic provides an estimate of this distance via the loss, one can also compute it independently for simpler cases.
        %
        For the 1D Gaussian example, the Wasserstein distance has a closed-form (it reduces to the difference in means when both distributions are Gaussian with equal width), and for the jet observables one can calculate one-dimensional Wasserstein distances for each observable’s distribution individually.
        %
        One can also monitored the \textbf{Vincze–Le Cam (VLC) divergence} \kd{cite DBLP:journals/corr/abs-2009-10838, nishiyama2022relationstightboundssymmetric} between the unfolded (reweighted) generation and the truth.
        %
        The VLC divergence, sometimes referred to as the triangular discrimination, is a symmetric divergence measure defined for two probability density functions $p$ and $q$ as
        \[
            \Delta_{VLC}(p,q) \;=\; \frac{1}{2}\int_{-\infty}^{\infty}\frac{\big(p(x)-q(x)\big)^2}{\,p(x)+q(x)\,}\,\dd x\,.
        \]
        Lower values indicate closer agreement between distributions ($\Delta_{VLC}=0$ if and only if $p=q$).
        %
        This metric is particularly sensitive to differences in the tails of distributions and provides complementary check to the Wasserstein distance.
        %
        During training, one could periodically apply the current set of weights $g(z)$ to a large sample of simulation events and then evaluate $\Delta_{VLC}$ and the Wasserstein distance between this weighted sample and the truth sample (using the known truth in validation studies).
        %
        A decreasing trend in both metrics as training progresses signals that RAN is converging towards the correct reweighting.
        
        For instance, in the Gaussian example with moderate smearing, we observed the VLC divergence between the weighted simulation and truth drop rapidly in the first few dozen iterations and flatten out at a low value, indicating that the generator had found an effective weighting scheme.
        %
        Similar behavior was seen in the jet case: the combined six-dimensional weight optimization gradually improved the agreement in all observables.
        %
        By the end of training, the unfolded distributions for all six jet observables were nearly indistinguishable from the truth, as quantified by final Wasserstein distances on the order of a few times $10^{-2}$ and VLC divergence on the order of $10^{-3}$ (see \kd{cite Andreassen:2019cjw} for comparable benchmarks with iterative methods).
        %
        We include summary tables (Table~\ref{tab:comp_wass} and Table~\ref{tab:comp_vlc} in Section~\ref{sec:results}) comparing these divergence metrics for RAN versus other methods; the RAN implementation achieves competitive or superior scores, validating our training procedure.

        Crucially, the monitoring process also guards against the possibility of divergence or weight blow--up.
        %
        If at any point the training appears to destabilize (e.g. the critic loss oscillates wildly or the weights $\qty{g(z)}$ start to become extremely large for a small subset of events), we can intervene by either stopping early or introducing additional regularization.
        %
        In practice, thanks to the built-in regularizers (spectral norm, gradient penalty, dropout), such pathological behaviour was rare.
        %
        One symptom we might encounter in some trials was the appearance of a few very large weights in sparsely populated regions of phase space.
        %
        This is a known issue in unfolding problems.
        %
        When there are regions that have little simulated density but non--negligible data density, any method will assign large weights to the few simulation events in that region.
        %
        RAN is no exception;
        %
        nonetheless, the spectral normalization on the critic help limit how fast and how large those weights could grow.
        %
        If needed, one could impose an explicit cap on weights or an additional \(L_p\) norm term in the loss to penalize overly large weights, but we did not find this necessary for our benchmarks.

    In summary, the ML implementation of RAN described above proved to be robust and reproducible across the different datasets.
    %
    By carefully specifying the architecture (with sufficient complexity to learn the needed reweighting, but regularized to avoid overfitting), using a stable training objective (Wasserstein GAN with gradient penalty) and optimizer (RMSProp), and employing diligent monitoring/early--stopping criteria, we ensured that the training converges reliably to a good solution.
    %
    The design choices were informed by both theoretical considerations (e.g. the advantage of WGAN for non--overlapping supports) and practical experimentation (e.g. the choice of 3 critic updates per 2 generator updates for efficiency).
    %
    The resulting RAN model can be straightforwardly re--implemented in common deep learning frameworks, and our reported hyperparameters can serve as a baseline for future applications.
    %
    All training scripts and data have been made available open--source to facilitate verification and reuse.
    %
    Thus, this implementation satisfies the dual goals of performance and transparency, which are essential for trustable unfolding in high--energy physics.

\section{Results}
    \label{sec:results}
    We evaluate the Reweighting Adversarial Network (RAN) on two case studies: a targetting one--dimensional Gaussian unfolding task and a realistic jet substructure unfolding problem.
    %
    In each case, RAN’s performance is compared to two baseline methods, the \textsc{OmniFold} algorithm~\kd{Andreassen:2019cjw,Andreassen:2021zzk} and Iterative Bayesian Unfolding (IBU)~\kd{DAgostini:1994fjx}.
    %
    We report quantitative divergence metrics for each method and analyse their ability to reproduce the true (particle--level) distributions (closure tests), as well as the methods’ stability under statistical fluctuations.

    \subsection{Gaussian Model with Smearing}
        We first consider a targeted unfolding scenario to stress-test RAN in a regime of limited detector--level overlap.
        %
        In this model, the truth particle--level distribution is a Gaussian $T \sim \mathcal{N}(0,,1)$, while the initial simulation (prior) distribution is $G \sim \mathcal{N}(-1,,1)$.
        %
        This introduces a modest discrepancy at particle level (a mean offset of 1 unit).
        %
        We then simulate a detector response that scales each observable by a constant “smearing” factor.
        %
        As the smearing factor increases above unity, the detector--level distributions of the data (smeared truth) and simulation (smeared prior) become increasingly separated (with less overlapping support), even though the underlying particle--level distributions remain fixed.
        %
        This controlled setup allows us to examine how each unfolding method copes with a progressively worsening mismatch between the observed (detector--level) data and the simulation.
        %
        Notably, methods like \textsc{OmniFold} rely reweighting at the detector level, which can fail if the detector--level supports do not overlap well.
        %
        In contrast, RAN performs reweighting before the detector simulation (at particle level) and uses an optimal transport--based loss to compare detector--level outputs.
        %
        We therefore expect RAN to be more robust in scenarios with poor detector--level overlap, provided the particle--level distributions still cover each other.

        To quantify performance, we use the Wasserstein\(-1\) distance ($W_1$) and the Vincze–-Le Cam (VLC) divergence $\Delta_{VLC}$ between the unfolded outcome and the true distribution as metrics of accuracy.
        %
        Figure~\kd{fig:omnifold-comp} summarizes the unfolding accuracy as a function of the smearing factor, with $\Delta_{VLC}$ plotted for RAN and \textsc{OmniFold}.
        %
        At minimal smearing (factor $1.0$, essentially an identity detector response), both methods perform excellently, and \textsc{OmniFold} achieves a slightly smaller divergence to truth than RAN (for example, at smearing $1$, $\Delta_{VLC}$ for \textsc{OmniFold} is $0.057\times10^{-3}$, compared to RAN’s $0.122\times10^{-3}$).
        %
        This confirms that when the detector effects are negligible and the problem reduces to a straightforward reweighting, \textsc{OmniFold}’s direct classifier--based approach is very effective---indeed, it finds a nearly perfect weighting in this easy regime.

        However, as the smearing increases and detector--level overlap deteriorates, \textsc{OmniFold}’s performance degrades sharply.
        %
        Beyond a smearing factor of about $1.5$, the \textsc{OmniFold} solution’s divergence from truth grows rapidly.
        %
        By the highest smearing values tested (where the detector--level simulation and data distributions are almost disjoint), \textsc{OmniFold}’s unfolded result significantly deviates from the true distribution (nearly an order of magnitude worse $\Delta_{VLC}$ compared to the no--smear case).
        %
        This is consistent with expectations.
        %
        \textsc{OmniFold} tried to reweight at detector level, and when it cannot find common phase space between data and simulation, it struggles to assign meaningful weights.
        %
        In stark contrast, RAN maintains a stable performance across the entire range of smearings.
        %
        As shown in Figure~\kd{fig:omnifold-comp}, RAN's $\Delta_{VLC}$ remains approximately constant (with only slight variation within statistical error) as smearing increases.
        %
        Even at the largest smearing factors, RAN's divergence to truth is essentially the same as it was at zero smearing, indicating that RAN successfully unfolds the distribution despite scalar multiplication.
        %
        This behaviour can be attributed to RAN's methodological property of only reweighting at particle level, combined with its use of an optimal transport loss at detector level, which, unlike a classifier, does not require overlapping support to provide a meaningful gradient---the adversarial training can adjust weights to minimize the Wasserstein distance between the smeared simulation and data, even if those distributions are partially disjoint.
        %
        The error bars in Figure~\kd{fig:omnifold-comp} (representing 1$\sigma$ uncertainties from bootstrapping the experiment) confirm the statistical stability of each method.
        %
        RAN’s results are consistent across bootstrap resamples, and while \textsc{OmniFold}’s uncertainty also grows with smearing, the trend of degradation is clear beyond the error bars.
        %
        In summary, this Gaussian study demonstrates that RAN is robust to limited detector--level support, recovering the true distribution reliably in a single pass, whereas \textsc{OmniFold} requires sufficient detector--level overlap to perform optimally. This advantage of RAN becomes critical in more complex, realistic scenarios where detector effects can be large.
    \subsection{Jet Substructure Unfolding Results}
        We next assess RAN on a realistic high-energy physics unfolding task: correcting jet substructure observables from detector--level “measured” data to particle--level truth.
        %
        This example serves as a stringent test of RAN’s ability to handle multi--dimensional, non--Gaussian distributions with sharp features and long tails.
        %
        The observables chosen span a range of distribution shapes---from broad distributions to sharply peaked ones and those with kinematic cutoffs---providing a comprehensive testbed.
        %
        The unfolding task is to reweight the \textsc{Pythia} generation such that, after detector simulation, it matches the “data” distribution (here, the \textsc{Herwig} detector-level output), and then to compare the reweighted \textsc{Pythia} particle-level jets to the \textsc{Herwig} particle-level truth.
        %
        RAN is applied to learn a single weighting function $g(z)$ that assigns a weight to each \textsc{Pythia} event (based on its features) to achieve this alignment, as described in Secs.\ref{subsec:arch}–\ref{subsec:mlimplement}.
        %
        For comparison, we also apply \textsc{OmniFold}~\kd{Andreassen:2019cjw,Andreassen:2021zzk} and the classical IBU~\kd{DAgostini:1994fjx} to the same task.
        %
        Since IBU is a binned algorithm, we perform separate one--dimensional IBU unfolds for each observable using a fine histogram binning, to benchmark its performance on each projection of the data.

        \subsubsection{Overall performance}
            Figure~\kd{fig:particle-level-distribution} presents the unfolded particle--level distributions for each of the six jet observables, for RAN, IBU, and \textsc{OmniFold}, compared to the truth distributions.
            %
            Each panel shows the truth spectrum (\textsc{Herwig}, treated as “true” data) in solid blue, the generation in orange, the unfolded result from RAN as a black dashed line, and the unfolded result from \textsc{OmniFold} as a red dashed line, and the result from IBU as a green dashed line.
            %
            The lower ratio sub--panels display the unfolded/Truth ratio for each method, indicating how well each method closes the gap to truth across the range of each observable.
            %
            We observe that both RAN and \textsc{OmniFold} achieve good agreement with truth across all observables, a non--trivial accomplishment given the complexity of the distributions, but RAN consistently provides a closer match, particularly in regions that are challenging to unfold.
            %
            For example, in the jet mass distribution (Fig.\kd{fig:particle-level-distribution}, top-left panel), the \textsc{Pythia} simulation initially undershoots the probability in the high-mass tail and misplaces the peak.
            %
            After unfolding, RAN’s distribution closely follows the truth curve: it accurately reproduces the peak around $m\approx 20$ GeV and the long tail up to high masses. \textsc{OmniFold} also significantly improves the agreement, but its ratio plot reveals a slight residual bias in the tail (deviations of order 5–10\% from unity in the highest mass bins), whereas RAN’s ratio is within a few percent of unity throughout.
            %
            A similar pattern is seen for the $N$-subjettiness ratio $\tau_{21}$.
            %
            The generator (\textsc{Pythia}) initially produces a broader $\tau_{21}$ distribution than truth (\textsc{Herwig}), indicating an overestimation of two--prong substructure.
            %
            RAN’s unfolding narrows this distribution to align almost exactly with the truth shape, correcting both the peak and the tail of the $\tau_{21}$ spectrum; \textsc{OmniFold} moves in the right direction but leaves a noticeable difference (the \textsc{OmniFold} unfolded $\tau_{21}$ distribution remains slightly too broad, with the ratio to truth dipping below 1.0 in the peak region and rising above 1.0 at higher $\tau_{21}$).
            %
            For observables like the jet width $w$ and logarithmic groomed mass $\ln\rho$, which have sharply peaked distributions near zero and long tails, RAN again excels: it captures the steep drop-off and the tail behavior with high fidelity, whereas \textsc{OmniFold} shows small mismodeling in the intermediate region (for $w$) or tail (for $\ln\rho$).
            %
            The groomed momentum fraction $z_g$ is a particularly challenging variable because of its hard cutoff at $z_g=0.5$.
            %
            Generative or reweighting methods often struggle to reproduce such cutoff behavior precisely.
            %
            We find that RAN handles this boundary reasonably well
            %
            The unfolded $z_g$ distribution from RAN matches the truth both in the low--$z_g$ region and near the cutoff, correctly recovering the falling slope as $z_g \to 0.5^-$.
            %
            \textsc{OmniFold}’s result for $z_g$ is also reasonable but tends to slightly undercorrect near the cutoff (its ratio is a bit below 1.0 approaching 0.5, indicating a remaining deficit in events just below the cutoff).
            %
            Finally, the constituent multiplicity $M$ (number of particles in the jet) is an interesting case.
            %
            It is a discrete distribution with a long tail.
            %
            RAN succeeds in reducing the discrepancy at both the low-$M$ and high-$M$ extremes, yielding an unfolded distribution that tracks the truth across the full range.
            %
            \textsc{OmniFold} improves the multiplicity distribution in the bulk region but exhibits larger fluctuations in the very high-multiplicity tail (partly due to limited statistics there and the difficulty for the classifier to learn in sparse regions).
            %
            In summary, qualitatively, RAN’s unfolded spectra are virtually indistinguishable from the truth for all six observables, within statistical uncertainties, whereas \textsc{OmniFold} shows minor but noticeable deviations in certain challenging regions.
            %
            The ratio (closure) plots in Fig.\kd{fig:particle-level-distribution} highlight RAN’s better closure: the RAN/Truth ratio is closer to 1.0 (often within a few percent) across the phase space, while the \textsc{OmniFold}/Truth ratio deviates by up to 5–15\% in some bins (especially in the tails of $m$, $M$, $\tau_{21}$, and $\ln\rho$).

            To make these comparisons quantitative, we compute the Wasserstein\(-1\) distances ($W_1$) and VLC divergences between each unfolded distribution and the truth distribution for all methods.
            %
            Table~\kd{tab:comp-w} and Table~\kd{tab:comp-vlc} summarize these metrics for RAN, \textsc{OmniFold}, and IBU on each jet observable (lower values indicate better agreement with truth; the tables also include, for reference, the baseline distances for the Generation vs. truth and Data vs. truth distributions with no unfolding).
            %
            RAN achieves the smallest divergence from truth in every one of the six observables under both metrics.
            %
            In particular, RAN outperforms \textsc{OmniFold} by a substantial margin in those observables identified as challenging: for the jet mass $m$, $W_1$(RAN) is $3.35\times10^{-2}$ (in units of the table’s scale) compared to \textsc{OmniFold}’s $8.02\times10^{-2}$, a roughly factor--of--2 improvement.
            %
            For the logarithmic groomed mass $\ln\rho$, RAN’s $W_1$ distance is $0.20\times10^{-3}$, dramatically smaller than \textsc{OmniFold}’s $4.15\times10^{-3}$ – an order-of-magnitude reduction, indicating RAN has far better accuracy in the extreme tail of this distribution.
            %
            Similarly, for $z_g$, RAN’s $W_1$ is about $0.80\times10^{-3}$ vs. \textsc{OmniFold}’s $6.32\times10^{-3}$, reflecting how well RAN handled the cutoff region.
            %
            Even in observables where \textsc{OmniFold} performed strongly, such as jet width $w$, RAN still edges out a win ($0.59\times10^{-3}$ vs. $0.81\times10^{-3}$).
            %
            The VLC divergence results (Table~\kd{tab:comp-vlc}) tell a consistent story: RAN yields the lowest $\Delta_{VLC}$ for all observables, with \textsc{OmniFold} typically second--best.
            %
            For example, $\Delta_{VLC}(m)$ is 2.31 (in the table’s scaled units) for RAN, versus 5.06 for \textsc{OmniFold} and 5.11 for IBU.
            %
            In some cases the gap between RAN and \textsc{OmniFold} is negligible (jet width $w$: 1.29 vs 1.33), while in others it is significant (multiplicity $M$: 5.83 vs 6.32 for \textsc{OmniFold} and a much larger 16.07 for IBU, indicating IBU struggled with $M$; or $\tau_{21}$: 0.92 vs 7.91 for \textsc{OmniFold}, an order of magnitude difference).
            %
            We emphasize that IBU, being a binned, per--observable method, generally underperforms the unbinned methods here.
            %
            This is likely due to binning and statistical issues: the need to choose finite bin widths leads to information loss and large uncertainties in sparse regions (for $M$, the tail probabilities in high bins were not well--estimated, leading to an inflated divergence).
            %
            In contrast, RAN and \textsc{OmniFold} operate on unbinned data and can leverage the full event information, yielding superior precision.
            %
            It is noteworthy that RAN’s advantage is achieved while producing a single set of event--level weights that simultaneously correct all six distributions.
            %
            In other words, RAN (like \textsc{OmniFold}) inherently unfolds the joint multi--dimensional distribution of these observables without splitting the problem into one dimension at a time.
            %
            The fact that a single model can balance all observables and still attain better per-observable accuracy is a strong testament to the effectiveness of the adversarial reweighting approach.

            Beyond accuracy, stability and efficiency are important considerations for unfolding methods.
            %
            We examined the stability of the RAN training by performing multiple independent training runs and by bootstrapping subsets of the jet dataset.
            %
            The variations in the resulting unfolded distributions (and in the $W_1$/VLC metrics) were found to be small, on the order of a few percent, indicating that the RAN solution is robust to statistical fluctuations and the stochastic nature of training.
            %
            Likewise, \textsc{OmniFold}’s iterative procedure, when run on the same data with different initializations, converged to comparable results (though minor differences in the weights per iteration can occur, the final distributions were consistent within uncertainties).
            %
            IBU, being an analytic iterative method, showed negligible run-to-run variation for fixed binning.
            %
            However, all iterative methods raise the question of convergence and tuning.
            %
            \textsc{OmniFold} requires choosing a stopping criterion or number of iterations (too few iterations can under--correct, too many can overfit the statistical noise), and IBU similarly depends on the number of iterations (or an implicit regularization).
            %
            In our \textsc{OmniFold} implementation we used five iterations (the standard approach advocated in ~\kd{Andreassen:2019cjw}) which was sufficient for good performance; using more iterations does not notably improve the agreement and can introduce instability.
            %
            RAN avoids this ambiguity entirely, since it is a one--shot training.
            %
            In practice, we found RAN’s training to converge steadily without signs of overtraining (monitored via validation loss) and to reach a stable solution within a few dozen epochs.
            %
            In terms of computational cost, RAN’s single--pass adversarial training (using a Wasserstein GAN-like setup) was comparable to the cost of a single \textsc{OmniFold} iteration.
            %
            But since \textsc{OmniFold} required two classifier trainings per iteration (one at detector level, one at particle level) and we performed two iterations (four trainings total), the overall runtime for RAN was roughly a fifth of that for \textsc{OmniFold} on this problem.
            %
            IBU is extremely fast for one--dimensional histograms, but extending IBU to many dimensions would be combinatorially expensive (and practically impossible for high-dimensional continuous observables), whereas RAN and \textsc{OmniFold} scale gracefully to high--dimensional data.
            %
            Thus, RAN offers an attractive trade--off: it achieves equal or better accuracy than \textsc{OmniFold} on these benchmark tasks, while being simpler to use (no iterative loop to manage) and potentially more efficient.

    \subsubsection{Generality of results}
    Combined with the findings from the Gaussian experiments, these jet experiments suggest that RAN’s advantages are not specific to a particular distribution but rather reflect general features of the algorithm.
    %
    RAN’s ability to handle non-overlapping detector effects (demonstrated in the Gaussian toy study) implies it could be deployed in experimental scenarios with poor detector resolutions or acceptance gaps, where traditional unfolding might struggle.
    %
    The jet study confirms that RAN scales to complex, realistic tasks, providing high--fidelity unfolding for multiple correlated observables in a single training.
    %
    We expect that these properties would carry over to other unfolding applications.
    %
    For example, measurements of other final states or higher--dimensional distributions (such as multi--variate phase--space distributions) should similarly benefit from RAN’s unbinned, non--iterative approach.
    %
    Of course, some caution is warranted when extrapolating beyond the tested cases: real experimental data may involve additional complexities like non--identical simulation vs. reality in ways not captured by generator differences alone, or require careful treatment of systematic uncertainties in the unfolding procedure.
    %
    However, the closure tests performed here (using one simulator’s data as “pseudo-data” and another’s as truth) are a stringent validation, and RAN passes them with flying colours.
    %
    In particular, RAN’s strong performance on observables with sharp features ($z_g$ cutoff) and heavy tails ($m$, $\ln\rho$) bodes well for its application to other distributions where similar challenges arise (e.g. distributions with kinematic edges or long perturbative tails).
    %
    Moreover, RAN’s inherently multivariate nature means that, unlike IBU, it can preserve and unfold correlations between observables, an important aspect for modern high--dimensional analyses in particle physics.
    %
    In summary, the results presented in this section demonstrate that RAN achieves state--of--the--art unfolding performance on both simple and complex tasks.
    %
    It matches or exceeds the accuracy of established methods like \textsc{OmniFold} and IBU, while offering improved stability in difficult scenarios and simplifying the unfolding workflow.
    %
    These characteristics make RAN a promising tool for future precision measurements and exploratory studies in which unfolding high--dimensional data without bias is critical.