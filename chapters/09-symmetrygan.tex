\chapter{Symmetries in Data: Connections to Unfolding Challenges}
\label{chap:symmetrygan}
\begin{itemize}
    \item The role of symmetries in physics and measurement processes
    \item Statistical definition of dataset symmetries
    \item SymmetryGAN: Discovering symmetries with adversarial learning
    \item Application dijet events
    \item Using symmetry knowledge to constrain unfolding problems
    \item Symmetry-aware unfolding for improved measurement precision
\end{itemize}
\section{Symmetries and Unfolding}
    \subsection{The Complementary Nature of Symmetry Discovery and Unfolding}
        Unfolding, as discussed, refers to the inverse problem of inferring underlying truth--level distributions from observed detector--level data, accounting for distortions due to limited resolution and acceptance.
        %
        Symmetry discovery, aims to identify invariant transformations of the data, that is to say, operations under which the probability distribution of the dataset remains unchanged in a statistical sense\kd{cite}.
        %
        At first glance, these two tasks appear distinct;
        %
        one concerns recovering numerical distributions, while the other uncovers structural invariances.
        %
        However, symmetry discovery and unfolding are in fact complementary facets of data--driven inference, and integrating the two can yield deeper insights and improved measurements.
    
        From a conceptual standpoint, both tasks share the common goal of revealing hidden truth from observed data.
        %
        Unfolding endeavours to remove the ``detector mask" and expose the true differential cross section or underlying distribution that generated the measurements.
        %
        Symmetry discovery seeks to reveal underlying structures or invariances in the data-—patterns that persist under transformations, reflecting fundamental symmetries of the physical process or the measurement apparatus.
    
        In practice, these goals are intertwined.
        %
        If one discovers a symmetry in the dataset, that knowledge can constrain the unfolding procedure by reducing the effective degrees of freedom in the solution space.
        %
        Conversely, a properly unfolded distribution is expected to manifest latent symmetries that may have been obscured by detector effects in the raw data.
        %
        Thus, identifying a symmetry and unfolding a distribution reinforce one another.
        %
        The former provides a guiding principle or constraint for the latter, while the latter provides a cleaner canvas on which the former can be observed.
    
        Imposing a symmetry as a prior constraint in unfolding can be seen as a form of physically motivated regularization.
        %
        For example, architectures that conserve four--momentum or enforce Lorentz invariance by design, as discussed in \cref{chap:ml-for-unfolding}, effectively impose such constraints\kd{cite}, narrowing the set of viable solutions.
        %
        By restricting solutions to those that respect a discovered or expected invariance, one reduces the space of admissible unfolded distributions to those that are physically plausible, which leads to improved stability and fidelity of the results\kd{cite}.
    
        This principle has been implicitly utilized in classical unfolding and simulation--based calibration.
        %
        For example, one could assume certain symmetries such as isotropy or detector uniformity when designing the response matrix or when combining symmetric regions of phase space to reduce uncertainties.
        %
        With a data--driven symmetry discovery tool in hand, one need not rely solely on presumed symmetries.
        %
        Instead, one can verify them empirically or even discover unexpected invariances.
        %
        In turn, these empirically verified symmetries can be fed back into the inference pipeline to sharpen measurements, such that a discovered symmetry can inform the unfolding algorithm so that the final measured distribution upholds the invariance.
    
        It is instructive to compare symmetry discovery and unfolding side by side to appreciate their complementary roles.
        %
        \cref{tab:unfolding_vs_symmetry} summarizes the differences and points of contact between the two.
    
        While unfolding typically requires an explicit model of the measurement process\footnote{e.g. a response matrix or a parametrized detector simulation} and often relies on supervised learning or iterative inversion techniques, symmetry discovery can be pursued in an unsupervised manner, requiring only the dataset and a class of transformations to probe.
        %
        The outcome of unfolding is a corrected distribution intended for direct physical interpretation.
        %
        The outcome of symmetry discovery is a characterization of invariances, a set of transformations $T$ such that the dataset’s distribution is invariant under $T$ within statistical uncertainties.
        %
        These outcomes are different in nature, but they are mutually beneficial.
    
        Knowledge of invariant structure can guide numerical inference, and conversely, obtaining a more accurate numerical distribution makes it easier to discern subtle invariant patterns.
        %
        In essence, symmetry discovery and unfolding form a feedback loop in the broader endeavour of measurement and inference, where each can improve the other.
        \begin{table}
            \centering
            \caption{Comparison of Unfolding and Symmetry Discovery}
            \label{tab:unfolding_vs_symmetry}
            \begin{tabular}{m{0.1\linewidth} | m{0.4\linewidth} m{0.4\linewidth} }
                \toprule
                 & \textbf{Unfolding} & \textbf{Symmetry Discovery} \\
                \midrule
                \textbf{Goal} & Reconstruct true distribution from observed data & Find transformation(s) which keep the distribution is invariant. \\
                \midrule
                \textbf{Input} & Measured data and detector response model & Measured data and a class of candidate transformations.\\
                \midrule
                \textbf{Output} & Unfolded differential cross section or probability density,  & Symmetry transformations or invariance properties. \\
                \midrule
                \textbf{Physics} & Incorporates known physics via priors or constraints (e.g. smoothness, conservation laws) to regularize solutions. & Incorporates generic transformation families (e.g. rotations, permutations) but does not assume a particular symmetry \textit{a priori}\\
                \midrule
                \textbf{Relation--ship} & Produces clearer distribution for true symmetries should manifest, enabling validation or discovery of invariances. & Provides constraints that can regularize unfolding by restricting solution space to symmetry-respecting distributions. \\
                \bottomrule
            \end{tabular}
        \end{table}

    \subsection{Symmetry--Aware Cross Sections}
        Differential cross section measurements lie at the core of particle physics.
        %
        They quantify how often certain processes occur as a function of kinematic variables (such as angles, energies, or invariant masses), serving as detailed tests of theoretical models.
        %
        Achieving high precision in these measurements is essential, as even subtle deviations between the measured spectra and theory predictions can signal new physics or the need for refined models.
        %
        In this context, symmetries play a pivotal role in both the design and interpretation of cross section measurements.

        Many physical processes come with known symmetry expectations.
        %
        For instance, in proton–-proton collisions producing particle jets, one expects azimuthal symmetry about the beam axis, i.e. the physics is invariant under rotations in the plane perpendicular to the beam.
        %
        Consequently, the differential cross section should, after correcting for detector non--uniformities, be independent of the absolute azimuthal angle $\phi$ of a jet or dijet system\kd{cite}.
        %
        Likewise, for processes initiated by identical colliding particles, one often anticipates a symmetry between forward and backward directions.
        %
        In a symmetric proton–-proton collider, this implies that the rapidity distribution of a centrally produced system \footnote{e.g. dijet pair} should be symmetric about zero rapidity\kd{cite}.
        %
        Such symmetry means that the cross section for producing a system at rapidity $+y$ is the same as at $-y$, all else being equal.
        %
        If a measured differential cross section exhibits a significant asymmetry in these variables after unfolding and acceptance corrections, it would either indicate a previously unaccounted detector bias or hint at a physical effect, both of which are of great interest to investigate.

        Being symmetry--aware in a measurement can mean two things.
        %
        First, verifying that expected symmetries are indeed present, within uncertainties, in the data, and second, leveraging those symmetries to improve the measurement.
        %
        On the verification side, symmetry considerations provide valuable consistency checks.
        %
        Experiments can test whether their unfolded distributions respect fundamental symmetries, and a failure to observe an expected symmetry is a red flag, prompting scrutiny of systematic effects or potential new physics contributions\kd{cite}.

        On the other hand, when a symmetry is confirmed, one can exploit it to gain statistical and systematic advantages.
        %
        For example, if a distribution is believed to be symmetric in a certain variable, one can ``augment" or combine data from symmetric regions, effectively doubling the effective statistics for that distribution.
        %
        A common practice is to report differential cross sections as a function of $|y|$ or other symmetry--reduced variables, which assumes $y \leftrightarrow -y$ symmetry and thereby reduces statistical fluctuations\kd{cite}.
        %
        By incorporating symmetry in this manner, uncertainties can be reduced and the measurement becomes more robust against localized fluctuations.

        Symmetry--aware analysis also serves to impose physically motivated constraints that guard against overfitting noise in the unfolding process.
        %
        If one has discovered that a distribution must be invariant under a transformation,\footnote{e.g. rotating the entire event by some angle, or exchanging two identical particles in the final state,} imposing this invariance in the unfolding procedure will tie neural network parameters or bin values together that would otherwise float independently.
        %
        This effectively decreases the number of free parameters describing the unfolded result, acting as a regularizer that prefers solutions consistent with the symmetry.
        %
        The net effect is an improvement in the precision of the measured cross section and a reduction in spurious oscillatory features that might arise from statistical fluctuations.
        %
        Moreover, by reducing the dependence on bins with low occupancy (because they are combined with their symmetric counterparts), binned symmetry-aware unfolding can also mitigate the impact of detector acceptance edges or inefficiencies in specific regions of phase space.

        It is important to note, however, that any symmetry--based constraint should be applied with careful consideration. One must ensure that the symmetry is either theoretically well-founded or empirically validated, lest one impose a false invariance and obscure a genuine asymmetry.
        %
        This caution further motivates the need for data--driven symmetry discovery and validation tools.
        %
        An experimenter can use methods like SymmetryGAN\kd{cite} to verify whether the data uphold the symmetry to a high degree of confidence.
        %
        Only then would one proceed to incorporate that symmetry into the unfolding process or in the presentation of results.
        %
        Thus symmetry-aware differential cross section measurements can harness known, or discovered invariances to enhance precision and reliability, while simultaneously providing a framework to detect symmetry violations that could point to new phenomena.
    
    The research presented in this chapter builds upon the foundations laid in earlier chapters of this thesis, extending the paradigm of symmetry utilization in the context of measurement and unfolding.
    %
    \cref{chap:theoretical-foundations}, in its survey of existing techniques, provides a statistical foundation for incorporating known symmetries into data analyses, for example, by augmenting jet images with rotations to enforce rotational invariance.
    %
    It also helps us note the challenges such approaches face, such as the need for smoothing and careful validation of assumptions.
    %
    \cref{chap:ml-for-unfolding} includes references to how \textit{a priori} known symmetries can be hard coded into machine learning models, introducing Lorentz group equivariant networks that guarantee Lorentz invariance in the unfolding of particle physics data\kd{cite}.
    %
    The inclusion of symmetry constraints in unfolding models described in \cref{chap:npu,chap:moment-unfolding,chap:ran} through preserving physical invariants like momentum or charge conservation in the generative model would reduce the solution space and lead to more physically plausible unfolded results\kd{cite}.
    %
    All of these strategies rely on prior knowledge of the symmetry.
    %
    This chapter shifts to an inference driven approach that provides a novel, flexible, and fully differentiable deep learning based method to discover symmetries directly from data using adversarial learning, which then might allow one to leverage those discovered symmetries to inform the unfolding process.
    %
    This perspective emphasizes the overarching theme of the thesis, the interplay of measurement and inference, by using data--driven insights in the form of symmetry discovery to enhance the core measurement task of differential cross section unfolding.

    The remainder of this chapter is organized as follows.
    %
    \cref{sec:formalism-and-role} introduces the formal statistical definition of a dataset symmetry, addressing subtleties like Jacobian volume effects via the concept of an inertial reference density.
    %
    It also provides a brief overview of the symmetries most relevant to HEP.
    %
    \cref{1} presents the SymmetryGAN framework, which employs a generative adversarial network to automatically learn symmetry transformations from data.
    %
    \cref{1} validates this approach on illustrative examples and then applies SymmetryGAN to simulated dijet events, demonstrating how it can uncover physically meaningful symmetries in collider data.
    %
    \cref{1} discusses how the discovered symmetry information can be used to constrain unfolding problems: we outline methods to incorporate symmetry constraints into the unfolding procedure to reduce uncertainties and bias.
    %
    In \cref{1}, the chapter introduces a symmetry aware unfolding methodology and discusses how enforcing the symmetries identified by SymmetryGAN can improve the precision of differential cross section measurements.
    %
    Finally, the chapter concludes by highlighting how the insights gained here connect back to the broader narrative of the thesis, reinforcing the benefits of combining machine learning driven discovery with principled measurement techniques.
\section{Formalism and Importance}
\label{sec:formalism-and-role}
    In physics and statistics, a symmetry refers to an invariance of a system or dataset under a well-defined transformation.
    %
    Formally, let $G$ be a group of transformations (continuous or discrete) acting on a space of states or observations $X$.
    %
    A physical system or probability distribution is symmetric under $G$ if applying any transformation $g \in G$ leaves the relevant observables unchanged.
    %
    In group--theoretic terms, there exists an action $g: x \mapsto g(x)$ such that for all $x \in X$ and $g \in G$, the value of an function $O(x)$ remains equal to $O(g(x))$.
    %
    However, if $p(x)$ denotes a probability density on $X$, a group $G$ is a symmetry of \(p\) if
    \[
        \forall g \in G \; p(x) \; \dd x = p(g(x)) \; \dd(g(x)).\kd{PhysRevD.111.072002}
    \]
    In measure--theoretic language, a symmetry corresponds to an invariant measure.
    %
    For any measurable subset $A \subseteq X$ and any transformation $g$, \(g\) is a symmetry of \(A\) if $\mu(A) = \mu(g\cdot A)$, meaning the measure assigned to outcomes in $A$ is the same as that for the transformed set $g\cdot A$.
    %
    This definition encompasses both continuous symmetries\footnote{Lie groups, such as rotations depending on a continuous angle parameter} and discrete symmetries \footnote{groups constructed as Jordan--H\"older extension\kd{Michael Aschbacher (2004)} e.g. a mirror reflection or a permutation of identical objects}.
    %
    Symmetry principles lie at the heart of modern particle physics and also strongly influence experimental measurements.
    %
    At the theoretical level, fundamental symmetries constrain the form of physical laws and often correspond to conserved quantities or selection rules.
    %
    At the data level, symmetries, and their breaking, shape the distributions of observed events and can be leveraged for more efficient data analysis.
    %
    In the context of colliders, many observables are governed by symmetries of the underlying theory as well as symmetries introduced by the detector and measurement process.
    %
    It is therefore crucial to articulate how these symmetries operate both in ideal physics scenarios and in real observations.
    %
    This section provides a rigorous overview of symmetries relevant to collider physics and measurements.
    %
    It begins with the fundamental symmetries in particle physics that underlie observable phenomena in \cref{subsec:hep-symmetries}.
    %
    \cref{subsec:detector-symmetries} discusses symmetries in detector response functions and how the measurement apparatus can preserve or violate underlying invariances.
    %
    Next, \cref{subsec:cross-section-symmetries} examines how symmetries manifest in measured cross sections and data distributions, clarifying the translation from physical symmetry to statistical patterns in experimental histograms.
    %
    Finally, \cref{subsec:noisy-symmetries} highlights the challenges in identifying symmetries from noisy data, setting the stage for data--driven symmetry discovery techniques.
    %
    This foundation will be essential for later sections that introduce methods like SymmetryGAN for learning symmetries from data.

    \subsection{Fundamental Symmetries in HEP}
    \label{subsec:hep-symmetries}
        Particle physics is built upon a framework of symmetries that determine the allowed forms of interactions and the conservation laws observed in experiments.
        %
        Spacetime symmetries, in particular subgroups of the Poincaré group, are foundational.
        %
        The Poincaré group includes continuous Lorentz invariance (rotations and boosts) and translations (in space and time).
        %
        Lorentz invariance implies that the laws of physics take the same form in any inertial reference frame.
        %
        Equivalently, physical observables can be expressed in terms of Lorentz-‐invariant quantities\footnote{e.g. invariant masses, angles, and dimensionless ratios} that remain unchanged under boosts or rotations.
        %
        For example, the Mandelstam variables $s$, $t$, $u$ in a scattering process or the decay angle distribution in a particle’s rest frame are formulated to respect Lorentz symmetry.
        %
        In practice, exact Lorentz invariance means there is no preferred direction or absolute velocity in the underlying theory.
        %
        A given collision process should yield identical outcomes whether the laboratory frame is, say, Earth--bound or boosted to a constant velocity.
        %
        As a consequence of Lorentz symmetry and spatial isotropy, angular momentum and linear momentum are conserved in isolated systems\footnote{Noether’s theorem associates these conservations with rotational and translational symmetry, respectively \kd{cite}}.
        %
        Additionally, time translation symmetry leads to energy conservation, ensuring that system's total energy and the collision centre--of--mass energy are fixed constants of motion.
        %
        These spacetime symmetries are exact symmetries of all known fundamental interactions and provide the basis for defining covariant formalisms in quantum field theory.

        Beyond spacetime, the internal symmetries of the Standard Model dictate the spectrum of particles and their interactions.
        %
        Chief among these is the gauge symmetry group $SU(3)_C \times SU(2)_L \times U(1)_Y$, which defines quantum chromodynamics and electroweak theory.
        %
        Gauge symmetries are local symmetries that require the introduction of gauge bosons;
        %
        although these are internal symmetries rather than symmetries of observable spacetime, they have observable consequences such as charge conservation, associated with $U(1)_Y$ hypercharge symmetry and electric charge, and the existence of multiple particle generations.
        %
        The gauge symmetries of the Standard Model are spontaneously broken in certain cases.\footnote{One of the most notable examples is the electroweak $SU(2)_L \times U(1)Y$ breaking to $U(1)_{\text{EM}}$ via the Higgs mechanism, which introduces masses for the $W^\pm$ and $Z$ bosons and differentiates the electromagnetic and weak interactions.}
        %
        However, even broken symmetries leave remnant effects, such as the custodial symmetry in the Higgs sector or approximate conservation of isospin in QCD.
        %
        These internal symmetries set selection rules.
        %
        Processes that violate gauge charge conservation are forbidden and decays proceed only via symmetric channels.

        Alongside continuous symmetries, several discrete symmetries play a crucial role in particle physics.
        %
        The most prominent are C (charge conjugation, exchanging particles with their antiparticles), P (parity, spatial inversion or mirror reflection), and T (time reversal).
        %
        Each of these can be considered a transformation that might leave the fundamental laws invariant.
        %
        In the Standard Model, CP symmetry is approximately a symmetry of electromagnetic and strong interactions, but notably broken in weak interactions.
        %
        This manifests as differences in the behaviour of matter and antimatter.
        %
        Like most notable instance of this is the well known CP violation in neutral kaon and $B$-meson decays means those processes occur at different rates or with different phase relationships than their CP-mirrored counterparts.\kd{}
        %
        If CP were an exact symmetry of the dynamics, one would expect, for example, the angular distribution of decay products in a mirror--reflected process, swapping particles for antiparticles, to be identical to the original.
        %
        The observed deviations are vital clues to physics beyond simple symmetries.
        %
        Parity by itself is also violated maximally in the weak interaction.\footnote{ classic examples are the left handed nature of neutrinos and the parity asymmetric angular distribution of electrons in polarized $^{60}$Co beta decay \kd{cite}.}
        %
        On the other hand, the strong and electromagnetic interactions conserve parity, so for many processes, especially at high energies where electroweak effects are subdominant, it is a good symmetry.
        %
        A collider process governed by QCD, like multijet production, should occur equally in a configuration and its mirror reflected image, unless the experimental setup selects a handedness.
        %
        Charge conjugation is likewise not a symmetry of the full Standard Model, since, for example, there are no right--handed neutrinos to pair with left--handed ones under C, but for purely electromagnetic processes C--symmetry implies, that producing a negatively charged particle is as likely as producing the corresponding positively charged antiparticle under equivalent conditions.
        %
        Importantly, the combination CPT is believed to be an exact symmetry of local quantum field theory.
        %
        CPT symmetry implies, for instance, that particle and antiparticle masses and lifetimes are exactly equal \kd{cite}.
        %
        While CPT is not directly tested by single distribution symmetries in colliders, it provides a fundamental consistency check on any observed CP or T violation. \cref{tab:fundSymSummary} summarizes these fundamental symmetries, their group-theoretic character, and their status in the Standard Model.
        \begin{table}
            \centering
            \caption{Summary of key fundamental symmetries relevant to collider observables.
            %
            ``Charge'' refers broadly to conserved / constrained quantities via Noether's theorem or selection rules.
            %
            “Status in the SM” indicates whether the symmetry is exact, approximate, or broken at tree level.
            }
            \label{tab:fundSymSummary}
            \begin{tabular}{m{0.12\linewidth} M M m{3cm}m{2cm}}
                \toprule
                \textbf{Symmetry} &
                \textbf{Group} &
                \textbf{Charge} &
                \textbf{Observables} &
                \textbf{Status in SM} \\
                \midrule
                \textbf{Lorentz} &
                    SO(3,1) &
                     x^\mu p^\nu - x^\nu p^\mu &
                    Invariant masses, angular distributions &
                    Exact \kd{cite} \\
                \textbf{Trans\-lation} &
                    \mathbb{R}^{1,3}&
                    p^\mu &
                    Missing–$p_T$&
                    Exact \kd{cite} \\
                \textbf{Gauge} &
                    \text{\footnotesize{$SU(3) \times SU(2) \times U(1)$}}&
                    \text{hypercharges} &
                    Color flow in jets; $W$ charge asymmetry; lepton universality &
                    Exact locally \kd{cite} \\
                \textbf{Electro\-weak breaking} &
                    \langle H\rangle \neq 0&
                    M_W, M_Z    &
                    \(M_W, M_Z, \frac{M_W}{M_Z}\) &
                    Broken\kd{cite} \\
                \textbf{C} &
                    q^\pm \leftrightarrow \bar{q}^\mp&
                    \alpha_{q^\pm} \;\text{vs.} \;\alpha_{\bar{q}^\mp}&
                    $\frac{e^+}{e^-}; \frac{\mu^+}{\!\mu^-}$ &
                    Conserved in EM/QCD; violated in weak \kd{cite} \\
                \textbf{P} &
                    \vec{x} \to -\vec{x} &
                    \text{handedness} &
                    lepton asymmetry in $\beta$-decay &
                    Conserved in EM/QCD; maximally violated in weak \kd{cite} \\
                \textbf{CP} & C+P&
                    \text{CKM matrix} &
                    $B$-meson asymmetry; electric dipole moments &
                    Approx. broken \kd{cite} \\
                \textbf{T} &
                    t \to -t &
                    \text{QFT amplitudes} &
                    EDM searches; K meson $T$-violation &
                    Broken if CP broken \kd{cite} \\
                    \textbf{CPT} &
                        C + P + T &
                        m, \tau_{q^\pm} \;\text{vs.} \;\tau_{\bar{q}^\mp}&
                        $m_{p}\!=\!m_{\bar p}$, $\tau_\mu\!=\!\tau_{\bar\mu}$ &
                        Exact \kd{cite} \\
                \textbf{Permutation} &
                    S_n/A_n &
                    \text{Bose/Fermi stats.} &
                    Jet ordering; boson correlation &
                    Exact \kd{cite} \\
                \bottomrule
                \end{tabular}
            \end{table}
            
            Beyond the Standard Model’s built in symmetries, there are approximate global symmetries that often prove useful in collider physics.
            %
            Examples include isospin symmetry, an $SU(2)$ symmetry treating up and down quarks as identical in the limit of equal masses, and flavour symmetries, like the $SU(3)$ of the light $uds$ quarks, which are not exact, but underlie patterns in hadron production and decay.
            %
            For instance, isospin symmetry implies that processes differing only by swapping an up quark with a down quark, such as producing a proton versus a neutron, have nearly equal cross sections, up to corrections from the up--down mass difference or electromagnetic effects.
            %
            Similarly, the universality of physical laws under interchange of identical particles leads to permutation symmetry.
            %
            If two particles of the same type appear in a final state, the probability distribution is invariant under exchanging them.
            %
            In quantum terms this is enforced by (anti)symmetrization of identical particle states.
            %
            In collider observables, permutation symmetry means that one cannot physically distinguish, say, which of two identical jets in an event is “jet 1” or “jet 2”---any labelling is arbitrary and the underlying physics treats the two jets on equal footing.
            %
            When calculating cross sections, this symmetry is accounted for by dividing by the a symmetry factor to avoid over counting identical configurations.
            %
            We will see that in data analysis one often has to impose an ordering, such as “leading” and “subleading” jet by momentum, for convenience, but the fundamental permutation invariance implies that any physical conclusion should not depend on this arbitrary ordering.

            In summary, fundamental symmetries, Poincaré (Lorentz and translations), gauge invariances, and discrete symmetries like C, P, CP, as well as permutation invariance for identical particles, provide a set of invariance principles for particle interactions.
            %
            These symmetries constrain the form of theoretical cross sections and transition rates.
            %
            Many measurable quantities in colliders such as cross sections, angular distributions, etc., either reflect these symmetries, when they hold, or provide avenues to detect symmetry breaking when deviations from the expected invariant patterns are observed.
            %
            However, the symmetries of nature at the fundamental level are not always manifest in what detectors actually record.
            %
            We next turn to how the detector response and measurement process can modify or obscure these symmetries.

    \subsection{Symmetries in Detector Response Functions}
    \label{subsec:detector-symmetries}
        A detector response function $r(x|z)$ describes the probability of observing a measurement outcome $x$ given a true particle-level state $z$.
        %
        This encapsulates effects of limited efficiency, acceptance, and resolution of a detector.
        %
        An ideal detector with perfect coverage and resolution would preserve all physical symmetries present at the particle level.
        %
        In reality, detectors often break or reduce symmetries that the underlying physics possesses.
        %
        Understanding which symmetries are preserved, approximated, or lost in $r(x|z)$ is crucial for interpreting measured data.
        %
        Here we discuss several common invariances and how they are affected by realistic collider detectors in their response.

        \subsubsection{Spatial Uniformity and Rotational Symmetry}
            Most collider detectors are designed with a roughly cylindrical geometry around the beam axis, aiming for azimuthal symmetry.
            %
            Hence they provide close to uniform coverage in the plane around the beam.
            %
            In an ideal scenario, if the physical process yields a uniform distribution in the azimuthal angle $\phi$ (i.e. no preferred direction around the beam line), a perfectly symmetric detector would register an equal number of events in each azimuthal segment.
            %
            In practice, small asymmetries creep in.
            %
            For example, the detector may have support structures or cabling at certain angles, or irregular segmentation, leading to variation in efficiency with $\phi$.
            %
            The electromagnetic calorimeter (ECAL), as an illustration, might be segmented into modules that cover specific $\phi$ slices.
            %
            Given this, events falling into the gap between modules could be recorded with lower efficiency or energy resolution, creating a slight $\phi-$dependence in the observed data even if the true distribution was uniform.
            %
            Detectors often have periodic segmentation, meaning continuous rotational invariance is broken down to a discrete rotation symmetry, invariant only under rotations corresponding to full module spacings.
            %
            As a concrete example, imagine a detector with 360 identical modules each covering $\Delta\phi = 1^\circ$.
            %
            This detector is invariant under rotation by 1--degree increments, but a rotation by an arbitrary angle (say $0.5^\circ$) would lead to a different alignment of a particle’s trajectory with respect to module boundaries, yielding a measurably different response.\kd{PhysRevD.111.072002}
            %
            Thus, the continuous symmetry is lost to a discrete one, and even that discrete symmetry may be imperfect if modules are not exactly identical or have time varying efficiency.
            %
            Thus azimuthal symmetry at the physics level is usually preserved approximately by detector design, but slight non-uniformities in $\phi$ response are common and must be accounted for either via calibration or acceptance corrections.

        \subsubsection{Polar Coverage and Boost Invariance}
            Collider detectors also have limited coverage in the polar direction, along the beam axis.
            %
            No real detector covers the full $4\pi$ solid angle;
            %
            there is always a cut--off at some polar angle (or pseudorange $\eta$) beyond which particles escape detection.
            %
            In particular, the forward regions close to the beam are notoriously difficult to instrument.
            %
            This breaks the full spherical symmetry of space.
            %
            A process that is symmetric under arbitrary rotations, such as a perfectly isotropic decay in its rest frame, will not appear isotropic in the laboratory measurement if a significant portion of the solid angle is unobserved.
            %
            Detectors are typically symmetric under rotations about the beam axis but not under arbitrary rotations that tilt the beam axis, because the beam direction is a fixed axis of symmetry.
            %
            In effect, the presence of incoming beams singles out a preferred direction, the beam axis $\hat{z}$, and detectors are built around this axis. Consequently, the data may reflect cylindrical symmetry, invariant under \(SO(2)\) rotations around $\hat{z}$, but not full $SO(3)$ rotational symmetry.
            %
            This also connects to Lorentz boost invariance.
            %
            While the underlying physics is Lorentz invariant, the detector is a fixed apparatus in one frame.
            %
            A boost along the beam direction i.e. a change of reference frame moving with respect to the collision will generally change how events are distributed relative to the detector acceptance.
            %
            For instance, consider a boost that causes particles to have higher longitudinal momentum;
            %
            in the lab frame, more particles will end up at small polar angles, closer to the beam line, where detection efficiency is lower, thus the observed distribution of, say, pseudorapidity $\eta$ will shift.
            %
            The detector has a finite acceptance in $\eta$, so a Lorentz boost that moves events into the far forward region will result in a fraction of events being lost.
            %
            Therefore, the measured distributions are not invariant under Lorentz boosts, even though the underlying parton--level kinematics can be expressed in Lorentz--invariant terms.
            %
            Thus the physical construction of detectors break global translational and boost symmetry by virtue of being static and having edges.
            %
            the data in the lab frame privileges the specific frame in which the detector is at rest.
            %
            A high energy collision viewed in a different inertial frame is physically identical, but the detector at rest will record it differently unless one corrects for acceptance and inefficiencies.

        \subsubsection{Resolution Effects and Approximate Invariance}
            Even if a symmetry could hold in principle, the resolution and threshold effects of detectors often spoil exact invariance.
            %
            A salient example arises with Lorentz invariance and invariant mass reconstruction.
            %
            As a thought experiment, imagine a two--body decay producing a pair of muons, such as $Z^0 \to \mu^+\mu^-$.
            %
            The true invariant mass of the muon pair is fixed, irrespective of the $Z$ boson’s momentum, because this is simply a Lorentz scalar.
            %
            However, a detector measures muon momenta with finite precision, and that precision typically degrades at high momentum\footnote{tracking detectors determine momentum from curvature in a magnetic field, which becomes very small for high momentum muons, leading to larger relative uncertainty in the measurement.}.
            %
            If a $Z$ boson is produced nearly at rest in the lab, its decay muons have moderate momenta and the detector might reconstruct the invariant mass with a narrow resolution.
            %
            If instead a $Z$ is produced with a large boost, the muons each have higher lab--frame momenta, and the detector’s momentum resolution broadens the reconstructed mass distribution.
            %
            The result is that the distribution of reconstructed $m_{\mu\mu}$ for boosted $Z$ events is broader (and potentially biased) compared to that for non--boosted events.
            %
            Thus, a Lorentz boost, which should not matter to an ideal measurement, actually changes the statistical distribution of an observable due to detector response.
            %
            This is an example of an approximately respected symmetry.
            %
            At low boost the symmetry holds well, but at high boost the symmetry is effectively broken by detector effects.
            %
            Similarly, thresholds in detector sensitivity (e.g. a calorimeter that only records energy above some minimum) can break symmetry under transformations that redistribute energy.
            %
            A detector that is be equally efficient for electrons and positrons, suggesting C-symmetry in detection.
            %
            However, if a process produces a wide spectrum of energies, a cut on low--energy particles could cause a difference.
            %
            More low-energy $e^+$ are likely to be lost than $e^-$ due to different interaction rates with material.
            %
            In such cases, even a symmetry of the physics might not result in equal measured counts.

        \subsubsection{Detector Mirror and Charge Symmetry}
            Detectors are not usually built to be fully symmetric under parity inversion or charge conjugation, even though we often assume these symmetries for the relevant physics should reflect in data.
            %
            A parity inversion would swap what we call “forward” and “backward” directions in the detector.
            %
            If the detector has identical coverage in the forward ($+z$) and backward ($-z$) hemispheres, one could say it is parity--symmetric with respect to the interaction point.
            %
            Many detectors strive for this by having symmetric endcaps on both sides of the interaction region.
            %
            However, even then, subtle asymmetries can exist because it is not feasible to prevent one side from having a slightly different material distribution or a different calibration from the other.
            %
            As a result, a process that is forward-backward symmetric in physics\footnote{In $pp$ collisions at the LHC, the two beam directions are equivalent so the distribution of particles as a function of rapidity $y$ should be symmetric about $y=0$} might show a small forward-backward asymmetry in the raw data if.
            %
            Experiments typically correct for such differences by equalizing calibrations, but the intrinsic detector response can break the symmetry.
            %
            Likewise, charge conjugation symmetry in detection would mean the detector is equally sensitive to positive and negative charges.
            %
            While the detector electronics and geometry generally don’t prefer one charge sign, magnetic fields introduce a notable asymmetry, because charged particles bend in opposite directions, and this can lead to charge--dependent acceptance.
            %
            In a magnetic spectrometer, positive particles bend outward in one direction and negatives in the opposite.
            %
            If the acceptance boundaries, like the edge of the detector volume, cut off tracks in one curvature direction more than the other, one will observe a difference in detection rates for $+$ vs $-$ even if production is symmetric. %
            Another example is that the different interaction of $e^+$ and $e^-$ with matter could lead to slightly different detection efficiencies.
            %
            These are second-order effects, but they illustrate that a detector is a physical object that need not respect the abstract symmetries of the theory.
            %
            Careful simulations and calibrations are performed to quantify and mitigate these asymmetries in collider experiments.

        \subsubsection{Permutation Symmetry and Identical Particles}
            A subtle aspect of detector response is its effect on permutation symmetry between identical particles in an event.
            %
            Physically, as noted, swapping two identical particles should change nothing in an ideal measurement.
            %
            Detectors, however, could introduce differences.
            %
            Two identical particles (say two photons) that go into different regions of the detector can have their energies might be measured with different resolutions or one might pass quality cuts and the other fail due to region--specific noise. 
            %
            s a result, the joint distribution of the two--particle system in the measured data might not be symmetric under exchange, even though it was at truth level.
            %
            As a simple example, consider two jets in an event where, at the particle level, the probability $P(E_1, \eta_1; E_2, \eta_2)$ is symmetric under $(1\leftrightarrow 2)$.
            %
            After detection, suppose jet 1 falls in the central barrel, with excellent energy resolution and jet 2 falls in the forward region, with poorer resolution and lower efficiency.
            %
            The measured energies $E_1^\text{(meas)}$ and $E_2^\text{(meas)}$ will have different response smearing.
            %
            If one then orders jets by measured energy and calls the highest $E$ jet the “leading” jet, the distribution of leading and subleading jet energy will not mirror one another exactly.
            %
            Effectively, the detector--induced asymmetry has assigned labels to the jets where none existed.
            %
            Analysts must be wary of these effects;
            %
            one option used is to ``symmetrize'' the analysis if possible to recover the permutation symmetry that the physics assures.

        \cref{tab:detectorSym} summarizes a few key examples of how an ideal symmetry at the particle--level can be broken or reduced by detector effects.
        %
        These examples illustrate why fully accounting for detector response is essential when testing physical symmetry hypotheses with data.
        
        \begin{table}
            \centering
            \caption{Ideal symmetry expectations versus typical detector-induced symmetry-breaking effects in collider experiments.}
            \label{tab:detectorSym}
            \small
            \setlength\tabcolsep{4pt}
            \renewcommand{\arraystretch}{1.15}
            \begin{tabularx}{\linewidth}{>{\raggedright\arraybackslash}p{0.25\linewidth} X X}
                \toprule
                \textbf{Transformation} &
                \textbf{Ideal Outcome} &
                \textbf{Detector Effect} \\
                \midrule
                Azimuthal Rotation &
                    Uniform event distribution in azimuthal angle $\phi$;
                    %
                    no preferred direction around the ring. &
                    Slight $\phi$ non--uniformity in measured data due to detector segmentation and gaps.
                    %
                    Only invariant under discrete rotations matching module periodicity.\\[2pt]
                Boost&
                    Physics unchanged under change of inertial frame;
                    %
                    kinematic distributions described in Lorentz-invariant terms. &
                    Detector at fixed orientation not boost-invariant.
                    %
                    Boost pushes more particles into forward regions, worsens some resolutions.
                    %
                    Observed distributions depend on the detector rest frame.\\[2pt]
                Parity&
                    If interaction is P--symmetric, processes occur equally in mirrored coordinates. &
                    Detector not mirror--symmetric.
                    %
                    Differences between $+z$ and $-z$ hemispheres introduce forward–-backward asymmetries in measured yields;
                    %
                    calibrations are required.\\[2pt]
                Charge Conjugation &
                    If physics is C--symmetric, particles and antiparticles produced at equal rates with identical kinematics. &
                    Charge asymmetric response is common.
                    %
                    Magnetic bending plus finite acceptance can cause differential detection.
                    %
                    Likewise $e^{+}/e^{-}$ or $\mu^{+}/\mu^{-}$ efficiencies can differ, biasing measured particle/antiparticle counts.\\[2pt]
                Permutation/Re-labelling&
                    Complete symmetry under exchange;
                    %
                    the joint distribution $P(z_1,z_2)$ is invariant if two identical particles' momenta or labels are swapped. &
                    Measured joint distributions can change upon swapping when the two objects land in detector regions with different responses.
                    %
                    A jet in the central region and one forward suffer different energy smearing than the opposite arrangement.
                    %
                    Imposing an arbitrary ordering (leading/sub=-leading) can hide the underlying symmetry.\\
                \bottomrule
            \end{tabularx}
        \end{table}
        Despite these challenges, experimentalists strive to design detectors with as much symmetry as feasible and to correct for known asymmetries.
        %
        For instance, collider detectors often have nearly full $2\pi$ azimuthal coverage and layered symmetries, segmenting in $\phi$ and $\eta$ uniformly, specifically to preserve rotational invariance and facilitate combining data over symmetric regions.
        %
        Detector simulation and calibration are used to quantify symmetry breaking.
        %
        If a $\phi-$dependence is observed in calibration data, it can be corrected so that the final analysis treats those variations as a systematic uncertainty or removes them.
        %
        Nonetheless, the reality remains that physical symmetries can fail to translate into measured symmetries.
        %
        In the language of probability distributions, if $p(z)$ is invariant under transformation $T$, but the response $r(x|z)$ is not invariant in the corresponding way, then the folded distribution $p(x) = \int r(x|z)\,p(z),\dd z$ will not be invariant under $T$ applied to $x$.
        %
        Only if both $p_{\text{truth}}$ and $r$ share the symmetry $T$ will $p_{\text{data}}$ exhibit it.
        %
        This conceptual understanding is vital when one interprets measured cross sections and tries to infer or discover symmetries from data.
        %
        One must always ask: is an observed symmetry or asymmetry coming from the physics, or from the detector?

    \subsection{How Symmetries Manifest in Measured Cross Sections}
    \label{subsec:cross-section-symmetries}
        Given the above considerations, one can now examine how symmetries and symmetry violations are reflected in the measured distributions that experiments record and report.
        %
        A measured cross section differential in some observable is effectively a statistical aggregate of many collision events, after selection cuts and corrections.
        %
        If the underlying physics possesses a symmetry, one might expect the differential cross section to reflect that, provided the measurement process does not hide or distort it.
        %
        In practice, one observes in histograms a mixture of genuine physical symmetry patterns and effects of detector acceptance or sample selection.

        \subsubsection{Exact Symmetries and Flat Distributions}
            A hallmark of a symmetry in a distribution is a repeated pattern indicating invariance.
            %
            For example, consider azimuthal invariance in a proton--proton collision.
            %
            Since the colliding protons provide a cylindrically symmetric initial state of two identical beams head-on, no physics process at the parton level prefers a particular $\phi$ direction.
            %
            Consequently, the true differential cross section $\frac{d\sigma}{d\phi}$ for an inclusive process is invariant under the transformation $\phi\mapsto\phi+\delta\phi$ (aside from small QED effects or residual detector magnetization influences).
            %
            If the detector has uniform $\phi$ coverage and the analysis has no $\phi-$dependent cuts, the measured distribution of events as a function of $\phi$ should be approximately flat.
            %
            Any significant deviation from flatness might indicate an instrumental problem or a selection bias.
            
            Once the symmetry has been established, one might combine data from all $\phi$ slices (since they are equivalent) to improve statistical precision, effectively using the symmetry to gather more data.
            %
            However, as noted, small modulations can appear if certain detector modules deviate from the rest; these are corrected or quoted as systematic uncertainties.
            %
            Another example is rapidities in symmetric collisions: in a $pp$ collider at equal beam energies, the center-of-mass frame coincides with the lab frame, and the process is symmetric under exchanging the two beam directions.
            %
            This implies that the distribution of particles in rapidity $y$ is symmetric about $y=0$ for processes that do not involve a bias (for instance, pure QCD dijet production should yield a symmetric $d\sigma/dy$ for jets, with equal activity in the forward ($+y$) and backward ($-y$) hemispheres).
            %
            This is why measurements of inclusive jet or hadron yields often present results as a function of $|y|$ or $|\eta|$ (absolute rapidity or pseudorapidity), invoking the symmetry $y \leftrightarrow -y$ to double the statistics and simplify presentation.
            %
            The physical symmetry (identical proton beams) justifies this, and one checks that, within uncertainties, the $+y$ and $-y$ distributions are consistent before merging.
            %
            Thus, a symmetry in initial conditions and dynamics (here, invariance under $y\to -y$) leads to a clear symmetry in the measured cross section (equal yields for $\pm y$).
            %
            If an unexpected asymmetry were observed, it would either signal new physics (e.g. a CP-violating effect or a bias in parton distribution functions) or, more likely, an issue like a mismodeled detector efficiency gradient.

        \subsubsection{Symmetries in Kinematic Shapes}
            Symmetries often impose recognizable shapes or constraints on distributions.
            %
            For example, energy and momentum conservation require that for each event, the vector sum of momenta of final state particles equals that of initial state.
            %
            As a result, distributions of total transverse momentum in events would be expected to peak at zero, and any significant imbalance indicates e.g. neutrinos or detector holes.
            %
            This is not a symmetry in the sense of a group acting on one event's space, but rather a deterministic constraint on the ensemble.
            %
            The distribution of missing momentum should be centered at zero and isotropic in azimuth.
            %
            Experiments can thus verify that the missing transverse momentum vector has no preferred direction to validate rotational symmetry and momentum conservation in aggregate.
            
            If one measures the transverse momentum spectrum of the first jet against that of the second jet in dijet events (with jets ordered by $p_T$), there is no fundamental reason for these spectra to differ except for the ordering bias.
            %
            The leading jet $p_T$ distribution will be harder by construction, and the subleading softer, any jet is equally likely to be at a given $p_T$ as its partner, aside from that ordering.
            %
            This symmetry can therefore by verified through the similarity between the distribution of subleading jet $p_T$ and the leading jet $p_T$ distribution of a lower--energy subset, or by symmetrizing the dataset by swapping jets event by event and seeing no change in overall two jet correlation distributions.
            %
            In summary, wherever a symmetry exists, one finds redundancies or equalities in the measured spectra: sections of phase space that should mirror other sections.
            %
            Experimental analyses often exploit this to measure detector backgrounds, by assuming that an uninstrumented region should have similar counts as a well-instrumented region after normalization.

        \subsubsection{Interplay of Physical and Detector Symmetries}
            It is important to disentangle which symmetries in a measured cross section come from physics and which from measurement procedure
            %
            An analysis might impose a cut that itself introduces a symmetry or asymmetry.
            %
            When presenting a measured cross section, unfolding detector effects to reconstruct particle level distributions to the extent possible, to report a cross section as it would appear with an ideal detector, can restore the symmetries that belong to the physics by removing the distortions of measurement\kd{cite}.
            %
            For example, if the raw data show a $\phi-$dependence due to detector inefficiency, the unfolded cross section vs $\phi$ should be flat (with larger uncertainties reflecting the correction).
            %
            In this sense, symmetries provide a consistency check.
            %
            If after unfolding one still sees a symmetry violation in a quantity that should be symmetric by physics, the unfolding procedure might be flawed.
            %
            Conversely, if a symmetry is expected to be broken by physics, one must be careful to ensure the detector is not distorting that asymmetry.
            %
            For instance, measuring a forward-backward asymmetry in top quark production, a sign of potential new physics or weak-interaction interference, requires excellent control of any detector differences between the forward and backward directions so that the observed asymmetry can be trusted as physical.

            A physical symmetry is a property of the underlying probability law.
            %
            It requires equal probabilities for events and their transformed versions.
            %
            A statistical symmetry of a dataset entails that the finite sample of observed data appears invariant under some transformation, within the limits of noise, so that with large data, one expects the symmetry to become apparent as the fluctuations average out.
            %
            If deviations persist significantly beyond expected fluctuations, that flags either a real symmetry violation or unaccounted systematics.
            

    \subsection{Challenges in Identifying Symmetries from Noisy Data}
    \label{subsec:noisy-symmetries}
        Identifying symmetries in experimental data is not always straightforward.
        %
        Noisy data, stemming from finite statistics, background processes, and detector imperfections, can obscure or mimic symmetry signals.
        %
        This section outlines the main challenges one faces in discerning true invariances or symmetry violations within collider datasets, and the need for methods like the SymmetryGAN approach developed later in this work to address these challenges.

        A fundamental challenge is that any empirical distribution has random fluctuations.
        %
        If an underlying distribution is perfectly symmetric (say truly uniform in $\phi$), a finite sample will still exhibit some variation across $\phi$ bins.
        %
        Hence any symmetry discovory method must have a mechanism to distinguish a real asymmetry from a mere fluctuation.
        
        Conversely, an underlying asymmetry can be washed out by limited statistics.
        %
        This is especially pertinent in searches for new symmetries or violations.
        %
        The signals are often at the level of small deviations and can be difficult to detect over statistical fluctuation.
        %
        Moreover, multiple comparisons increase the probability that one finds an apparent ``symmetric pattern" in some projection of the data purely by chance.

        As discussed, detector effects can induce or conceal asymmetries.
        %
        Often the largest uncertainties in measuring symmetry come from how well we understand the detector.
        %
        For example, in measuring a forward--backward asymmetry, uncertainties in the relative efficiency of the forward and backward region directly translate to uncertainty in the asymmetry observable.
        %
        If those efficiencies are poorly known, one might not be able to distinguish a symmetry violation from detector bias.
        
        Similarly, backgrounds, other processes that mimic the signal, might not share the symmetry of the signal.
        %
        Suppose one is looking for a symmetry in a certain particle decay distribution; if there is a significant background from a different process that does not respect that symmetry, the combined data will appear to break the symmetry even if the signal alone is symmetric.
        %
        Careful background subtraction or isolation is required.
        %
        In practice, identifying a symmetry often involves comparing two distributions (e.g. $P(x)$ vs $P(Tx)$ for some transformation $T$) and seeing if they differ. If they do, one must estimate if the difference is due to known systematic effects.
        %
        This typically demands high-precision calibration.
        %
        For instance, to confirm CP symmetry in production of particle vs antiparticle to the $10^{-3}$ level, one needs detector efficiencies known to better than 0.1\% between positively and negatively charged particle detection.\kd{cite}

        \subsubsection{Dimensionality challenges}
            Collider events are high dimensional, consisting of many particles with various kinematic attributes.
            %
            A symmetry might not be evident in any single one dimensional projection, but rather in a complicated combination of variables.
            %
            For example, Lorentz invariance is best seen when considering all four--momenta together or invariants like masses; a naive look at just one momentum component would not show it.
            %
            Permutation symmetry in a multijet event is a property of the joint distribution of all jet momenta, not necessarily obvious if one only looks at single--jet spectra.
           
            This is where machine learning methods become attractive, because they can, in principle, detect subtle patterns in high dimensional data.
            %
            However, even ML models need guidance.
            %
            The space of possible transformations is huge, and hence searching it naively for invariances is intractable.
            %
            Hence traditional methods often restrict attention to physically motivated symmetry transformations (rotations, reflections, boosts, particle exchanges, etc.).

            Since scanning for symmetries by comparing all possible pairs of transformed distributions is computationally prohibitive, as data volumes grow and analysis spaces become more complex, we need more automated symmetry discovery mechanisms.
            %
            The SymmetryGAN approach discussed in this thesis is one attempt to automate the discovery of symmetries by leveraging generative adversarial networks.
            %
            Conceptually, SymmetryGAN will trains a generator (applying candidate transformations) against a discriminator to test if the transformed data looks statistically identical to the original data.
            %
            If the generator generates a transformation under which the discriminator is maximally confounded, that transformation corresponds to a symmetry of the data distribution.
            %
            Implementing this is challenging: the model must search a continuous space of transformations, handle approximate symmetries, and avoid trivial solutions.
            %
            A careful choice of network architecture using known equivariances are needed to make such learning robust.\kd{cite}

            In summary, identifying symmetries from noisy collider data requires
            \begin{enumerate}
                \item Sufficient statistics and rigorous statistical tests to differentiate real invariances from fluctuations,
                \item Precise control of detector systematics to avoid mistaking detector effects for or against symmetry,
                \item Methods to probe high-dimensional and subtle symmetry patterns that might elude simple binned analyses, and
                \item Methodological consideration to handling approximate symmetries in a principled way.
            \end{enumerate}
            These challenges motivate the development of tools like SymmetryGAN, which I will introduce in the next sections.
            %
            Such tools aim to combine physical insight with machine learning's ability to detect patterns, thereby providing a statistical discovery framework for symmetries.
            %
            SymmetryGAN and similar approaches offer a promising path to unveil symmetries that are latent in complex data.
            %
            The rigorous understanding of symmetry and symmetry--breaking provided in this section will form the foundation on which those computational methods build, ensuring that any discovered ``symmetry" is physically meaningful and relevant to the challenges of unfolding and analyzing collider data.

\section{Statistical Definition of Dataset Symmetries}
    \label{sec:statistical-symmetries}
    The concept of symmetry in physics typically evokes images of rotational invariance in crystals, parity conservation in weak interactions, or gauge transformations in field theory.
    %
    Yet when we turn our attention to experimental data, especially the high dimensional datasets emerging from modern collider experiments, the notion of symmetry becomes surprisingly subtle.
    %
    What does it mean for a collection of measured events to possess a symmetry?
    %
    This question, deceptively simple in appearance, reveals profound connections between statistical inference, group theory, and the fundamental challenge of unfolding detector effects from observed data.
    \subsection{Distinction between point and dataset symmetries}
        Statistical symmetries in datasets represent a fundamental departure from traditional geometric symmetries, requiring careful consideration of probability measures, transformation Jacobians, and reference densities.
        %
        This section explores the mathematical foundations, practical applications, and machine learning approaches to understanding and leveraging dataset symmetries.
        %
        The distinction between symmetries of individual data elements and entire datasets lies at the heart of statistical theory of symmetries.
        %
        For individual data points, symmetry is characterized by simple invariance: a transformation \(g\) preserves element \(x\) if \(g(x) = x\). This represents a straightforward geometric notion where specific points remain fixed under transformation.
        %
        Distribution level symmetries, however, operate on probability measures rather than individual points.
        %
        A measure space \(X\) with probability measure \(\mu\) exhibits symmetry under group \(G\) when the measure remains invariant.\kd{}
        \[
            \forall A\subseteq X \, \forall g\in G\, \mu(A) = \mu(g(A))
        \]
        This measure theoretic definition captures the statistical properties of entire distributions rather than individual elements.
        
        The critical insight, as formalized in the SymmetryGAN framework, is that dataset symmetries are ambiguous due to Jacobian factors introduced during coordinate transformations.\kd{}
        %
        Unlike point symmetries, where transformations either preserve or don't preserve specific locations, dataset symmetries must account for how probability densities transform under coordinate changes.
        %
        This fundamental difference necessitates the introduction of inertial reference densities to properly define statistical symmetries.
    \subsection{Inertial reference densities and their theoretical role}
        The concept of inertial reference densities emerges as a necessary theoretical construct, analogous to inertial frames in classical mechanics.\kd{}
        %
        These reference densities provide a baseline against which statistical symmetries can be meaningfully defined, resolving the ambiguity inherent in coordinate transformations of probability measures.

        In the formal framework, a reference density \(\rho(x)\) establishes a canonical measure for comparing probability distributions, enabling the definition of relative entropy
        \[
            H_\rho[X] = -\mathbb{E}_\mu[\log\frac{\dd\mu}{\dd\rho}].
        \]
        Additionally, it provides a coordinate--independent way to specify symmetry transformations, ensuring that symmetry definitions remain consistent across different parameterizations of the same statistical manifold.
            %
        This can be analogized to phase space shifting operations that leave the Gibbs integration measure invariant can be understood as gauge transformations, with the reference density playing the role of a gauge fixing condition.\kd{Phys. Rev. Lett. 133, 217101 (2024)}
        
        The mathematical machinery for statistical symmetries centers on how probability densities transform under coordinate changes.
        %
        Under a transformation \(X = g(Y)\), probability densities transform as
            \[
                p_y(y) = p_x(g(y)) |\det(g'(y))|
            \]
        This transformation law, involving the Jacobian determinant \(|\det(g'(y))|,\) ensures probability conservation.
        %
        The Jacobian factor measures how volumes scale under transformation.
        %
        \(|\det(J)| = 1\) characterises volume preserving transformations like rotations and reflections.

        For a probability distribution to exhibit symmetry under group action \(G\), it must satisfy the condition
        \[
            \forall g\in G\,\forall x\in X\,p(x) = p(g(x))\,|\det g'(x)|
        \]
        This condition is far more restrictive than point symmetry, as it demands global consistency across the entire probability measure.

        The group-theoretic formulation provides additional structure.
        %
        A group \(G\) acts on a probability space \((\Omega, F, \mu)\) through measurable maps that preserve the \(\sigma-\)algebra structure.\kd{doi.org/10.1016/S1874-575X(06)80028-7}
        %
        Such a group can be decomposed using the orbit--stabilizer theorem, decomposing the action into orbits and stabilizers, and representation theory enables systematic construction of invariant functions and decomposition of function spaces into irreducible components.
        
    \subsection{Applications to particle physics}
        Particle physics provides a rich domain for applying statistical symmetries, particularly in detector calibration, data unfolding, and inference tasks at collider experiments.\kd{arXiv:2009.14613}
        %
        In detector response modeling, statistical symmetries constrain how identical particles must be treated. \kd{doi.org/10.17226/6045}
        %
        Permutation symmetry requires that detector analysis respect the indistinguishability of identical particles, affecting event reconstruction algorithms, particle identification methods, and background estimation techniques. \kd{doi.org/10.1073/pnas.93.25.1425}
        %
        Response matrices used in unfolding procedures must respect particle exchange symmetries, with regularization procedures preserving these symmetry properties.

        Unfolding represents an interesting application where statistical symmetries could guide the correction of detector effects.
        %
        The process must preserve the statistical symmetries of the underlying physics, maintaining correlations between identical particles and ensuring conservation laws implied by symmetries remain valid.
        %
        Full phase space unfolding could be facilitated by statistical symmetries that constrain unfolding procedures in high-dimensional phase spaces.
        %
        Background estimation methods also exploit symmetry properties through carefully designed control regions, while systematic uncertainties account for potential symmetry violations.

        In this way statistical symmetries fundamentally shape measurement and inference tasks through multiple pathways.
        %
        Phase space shifting operations that preserve physically meaningful quantities have even been used outside of HEP to develop new computational approaches for molecular simulations.\kd{doi.org/10.1038/s42005-021-00669-2}
        %
        These symmetries lead to exact correlation relations between forces and observable properties, offering systematic ways to derive new statistical relationships.
        %
        In machine learning too, encoding known symmetries dramatically improves performance.
        %
        Symmetry encoding reduces sample complexity exponentially, with multidimensional symmetries providing disproportionately large returns.\kd{2303.14269}
        %
        Equivariant neural networks, designed to respect known symmetries through architectural constraints, achieve superior data efficiency and generalization.\kd{2409.07327}

        The theoretical framework connects reference measures, symmetry principles, and statistical inference in profound ways.
        %
        Reference measure selection can be guided by symmetry considerations, leading to principled approaches for prior selection in Bayesian inference. \kd{DOI:10.2307/1968511}
        %
        Symmetry principles suggest natural classes of invariant estimators, while statistical analogs of physical conservation laws emerge from symmetry considerations, constraining the behavior of statistical systems.\kd{10.1073/pnas.93.25.14256}
        %
        This unified framework offers concrete advantages for computational statistics and machine learning while opening new directions for both theoretical development and practical applications.

\section{SymmetryGAN: Discovering Symmetries with Adversarial Learning}
\label{sec:symmetry-gan-main}
The challenge of automatically discovering symmetries in high dimensional data represents a fundamental problem at the intersection of physics and machine learning.
%
While physicists have long relied on theoretical insight to identify symmetries, the explosion of complex data from modern experiments demands automated approaches.
%
SymmetryGAN emerges as an elegant solution, leveraging the power of adversarial learning to discover hidden invariances without prior knowledge of their existence.

The core insight driving SymmetryGAN is deceptively simple.
%
If a transformation truly represents a symmetry of a dataset, then a well trained discriminator should not be able to distinguish between the original data and its transformed counterpart.
%
This principle, when combined with the rigorous framework of inertial reference densities developed in the previous section, yields a powerful methodology for symmetry discovery that is both theoretically grounded and practically effective.
    \subsection{The SymmetryGAN Architecture}
    The SymmetryGAN framework modifies the traditional generative adversarial network architecture in a fundamental way.
    %
    Rather than mapping from random noise to structured data, SymmetryGAN maps from data to data, learning transformations that preserve the underlying probability distribution.
    %
    Like a traditional GAN, however, SymmetryGAN consists of two neural networks engaged in adversarial training.
    \begin{enumerate}
        \item A generator \(g: \R^n to \R^n\) that learns symmetry tranformations, and
        \item A discriminator \(d: \R^n to [0, 1]\) that attempts to distinguish original from transformed data.
    \end{enumerate}
    The generator network \(g\) parametrizes potential symmetry transformations of the input space.
    %
    Unlike traditional GANs where the generator creates new samples, here it transforms existing data points.
    %
    The discriminator \(d\) plays its familiar adversarial role, attempting to classify whether a given sample comes from the original dataset or has been transformed by \(g\).

    The training objective is related but different from standard GANs.
    %
    The loss function is
    \[
        L[g,d] = -\frac{1}{N}\sum_{i=1}^{N}\left[\log(d(x_i)) + \log(1-d(g(x_i)))\right]
    \]
    This formulation differs from the usual binary cross entropy in that the same data samples appear in both terms, creating a self--adversarial structure.
    %
    The generator seeks transformations that maximally confuse the discriminator, while the discriminator learns to detect deviations from the true data distribution.

    By varying the loss with respect to the neural networks, one can show that at the theoretical optimum, the optimal discriminator is
    \[
        d_*(x) = \frac{p(x)}{p(x) + p(g_*(x))|g_*'(x)|} = \frac{1}{2}
    \]
    and the optimal generator obeys
    \[
        p(x) = p(g_*(x))|g_*'(x)|.
    \]
    i.e.the discriminator is maximally confused with equal probability of classifying any sample as original or transformed, and the generator obeys precisely the definition of a statistical symmetry.

    The architecture naturally handles both discrete and continuous symmetries.
    %
    For discrete symmetries like reflections or cyclic groups, the generator learns specific transformation matrices.
    %
    For continuous symmetries like rotations, it parametrizes the corresponding Lie group elements.
    %
    This flexibility makes SymmetryGAN applicable across diverse physical systems.

    \kd{fig:symmetrygan-architecture} shows the modified GAN architecture of SymmetryGAN.
    
    \subsection{Machine Learning with Inertial Restrictions}
    The incorporation of inertial reference densities, as established in \cref{sec:statistical-symmetries}, presents both theoretical necessity and practical challenges.
    %
    These restrictions can be implemented through three distinct approaches.
    \begin{enumerate}
        \item Simultaneous Discrimination:
            %
            In this approach, discriminators evaluate transformations on both the target dataset and samples from the inertial density.
            %
            The loss function is extended to include terms from the inertial distribution \(p_I\)
            \[
                L_{\text{sim}}[g,d,d_I] = L[g,d] + \lambda L_I[g,d_I]
            \]
            where \(d_I\) is a separate discriminator network specialized for the inertial distribution.
            %
            This method offers maximal flexibility, allowing discovery of symmetries even when the inertial distribution's symmetries aren't known analytically.
            %
            However, it requires the ability to sample from \(p_I\) which becomes problematic for improper priors like uniform distributions on \(\mathbb{R}^n\).
        \item Two--Stage Selection:
            %
            This approach involves first identifying all PDF--preserving maps, then filtering for those preserving the inertial density.
            %
            The two--stage approach decouples the symmetry discovery problem into training multiple generators to find PDF--preserving transformations of the target data and \textit{post hoc} verifying which transformations also preserve \(p_I\)
            %
            While conceptually clean, this method proves computationally wasteful, as the space of PDF--preserving maps vastly exceeds the space of true symmetries.
        \item Upfront Restriction:
            %
            Here one constrains the generator architecture to only produce transformations that preserve the inertial density by construction in the first place, using prior knowledge about \(p_I\).
    \end{enumerate}
    It is this third approach that is adopted in the SymmetryGAN implementation.
    %
    When the inertial distribution is uniform on \(\mathbb{R}^n\) the generator is restricted to equiareal maps---transformations with unit Jacobian determinant.
    %
    For linear transformations, this corresponds to the special linear group \(SL_n(\mathbb{R})\) and its extensions.

    The upfront restriction method offers computational efficiency gains by searching only among valid symmetrie in addition to theoretical guarantees that discovered transformations are true symmetries.
    %
    It also simplifies training without multiple discriminators or \textit{post hoc} verification.

    The restriction to linear equiareal maps, while limiting, captures a rich class of symmetries including rotations, reflections, shears, and all their compositions.
    %
    The Iwasawa decomposition\kd{} provides a systematic parametrization of \(g\in GL_n(\R)\).
    \[
        g = \sqrt{|d|} \cdot R(\theta) \cdot D(r) \cdot S(u) + v
    \]
    where \(d\) is the determinant, \(R(\theta)\in SO_n(\R)\) is a rotation, \(D(r)\) is a dialatation, \(S(u)\) is a sheer, and \(v\in\R^n\) is a translation.
    %
    As an illustration, in two dimensions,
    \[
        R(\theta) = \mqty[\cos\theta&\sin\theta\\-\sin\theta&\cos\theta] \quad D(r) = \mqty[r&0\\0&\frac1r] \quad S(u) = \mqty[1 & u\\0&1]\quad v = \mqty[v_1\\v_2]
    \]
    This decomposition enables efficient exploration of the symmetry space through gradient descent.

    \subsection{Deep Learning Implementation Details}
    The practical implementation of SymmetryGAN requires careful attention to architectural choices, training dynamics, and numerical stability.
    %
    The framework's success depends on balancing the adversarial training process while maintaining the theoretical properties that enable symmetry discovery.
    
    For linear symmetries, the generator directly parametrizes transformation matrices.
    %
    Rather than using fully connected layers, the generator outputs parameters of the Iwasawa decomposition, ensuring all generated transformations lie within the constrained search space.
    %
    This architectural choice provides interpretability, where each parameter has clear geometric meaning, stability, by avoiding ill-conditioned matrices through structured parametrization, and efficiency, by reducing the parameter count compared to unconstrained matrices

    For discovering specific symmetry subgroups, the loss function can further incorporate additional constraints.
    %
    For example, to find cyclic symmetries \(\mathbb{Z}_q\) the loss can be augmented to encourages \(g^q = \text{id}\).
    \[
        L_{\text{cyclic}}[g,d] = L_{\text{BCE}}[g,d] + \alpha \sum_{i} ||g^q(x_i) - x_i||^2.
    \]

    The discriminator architecture follows standard practices for the data domain.
    %
    For low--dimensional examples, simple feedforward networks with \(2-3\) hidden layers suffice.
    %
    †he discriminator need not be overly complex.
    %
    Its role is to provide gradient signal for the generator's symmetry search, not to achieve perfect classification.

    The optimization landscape analysis reveals its topological structure.
    %
    For simple distributions like Gaussians, the loss landscape contains distinct maxima corresponding to each symmetry, separated by deep valleys.
    %
    This topology explains why random initialization reliably discovers different symmetries.
    %
    The generator converges to the nearest local maximum, which corresponds to a true symmetry.

    The framework demonstrates remarkable robustness to hyperparameter choices.
    %
    Across experiments with Gaussian mixtures and particle physics data, a range of hyperparameter settings achieve consistent symmetry discovery.

    \subsection{Verification Protocols}
    A few different checks can be applied to verify that the discovered transformation represents a true symmetry.
    \begin{itemize}
        \item Visual inspection: Visually inspecting low dimensional slices of the original and transformed data serves as a simple sniff test.
        \item Loss value: True symmetries always achieve a loss of \(2\log2\) at convergence.
        \item Distribution matching: Statistical divergences between \(X\) and \(g(X)\) such as the KL divergence measure whether the two datasets have the same probability density or not.
        \item Invarian verification: known invariants, such as moments, should be identical between the distributions.
    \end{itemize}
    
    \subsection{Other Symmetry Discovery Methods}
        The landscape of symmetry discovery in machine learning has evolved considerably, with approaches ranging from classical statistical methods to modern deep learning architectures.
        %
        Understanding SymmetryGAN's position within this ecosystem illuminates both its unique contributions and its connections to broader methodological trends.

        Classical statistical approaches traditionally relied on hypothesis testing and moment matching.
        %
        These methods typically test specific, pre--defined symmetry hypotheses by relying on low-dimensional projections or summary statistics.
        %
        They struggle with high-dimensional data or complex symmetry groups and require extensive domain knowledge to formulate appropriate tests.

        The Cramer-von Mises and Anderson-Darling tests\kd{cite} exemplify this approach, checking whether transformed data follows the same distribution as the original.
        %
        While rigorous, these methods scale poorly and cannot discover unexpected symmetries.

        Early machine learning methods introduced automation to the symmetry discovery process.
        %
        Methods like Group-Invariant Autoencoders\kd{cite} learn representations invariant to known symmetry groups but cannot discover unknown symmetries
        %
        Spatial Transformer Networks\kd{cite} learn task-specific transformations but don't explicitly identify symmetries \textit{per se}.
        %
        Maximum Mean Discrepancy approaches\kd{cite} and other kernel methods can test symmetry but struggle with continuous groups.

        Contemporary deep learning methods have introduced a range of innovations to the field.
        %
        Lie Group Learning Networks directly parametrize Lie algebra generators\kd{cite}.
        %
        Such methods have excellent theoretical grounding, and physically interpretable parameters.
        %
        They are, however, restricted to continuous symmetries, and require complex optimization for convergence.
        %
        Latent LieGAN (LaLiGAN) pioneered the discovery nonlinear symmetries via latent space linearization\kd{cite}.
        %
        It enabled the field to make strides in handling nonlinear group actions, but the additional complexity leads to potential latent space artifacts
        %
        Reinforcement Learning enables automatic augmentation discovery through policy learning\kd{cite}.
        %
        The method is flexible in being task--performance driven, and it handles discrete symmetries well.

        SymmetryGAN exemplifies several important trends in modern machine learning for HEP.
        %
        Methods have broadly benefited from incorporating domain knowledge through architectural constraints and loss terms.
        %
        In general the rise of equivariant eearning has driven the move beyond invariance to discover the underlying symmetry structure.
        %
        SymmetryGAN also exemplies the trend towards interpretability by producing human-understandable symmetry parameters rather than opaque features.
        %
        The method is however limited in its restriction to parametrizable symmetry classes.
        %
        It also exclusively enables symmetry discovery, the task of identifying elements of the symmetry group of a dataset.
        %
        Although the paper does provide some steps towards symmetry inference, the task of identifying and classifying the symmetry group as a whole, it remains challenging to infer complete symmetry group structure from discovered elements.
        %
        The field continues to evolve rapidly, with SymmetryGAN representing a significant step toward automated, theoretically grounded symmetry discovery.
        %
        Its success in particle physics applications, detailed in the following section, demonstrates the practical value of this principled approach to uncovering hidden invariances in complex data.
        Gaussian mixture models demonstrate fundamental capabilities
